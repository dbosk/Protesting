There are some tasks that the organizers must accomplish prior to the protest 
itself.
For example, they must decide who are the most suitable candidates to attend 
the event, how to let them know about the protest and what preliminary 
information they should get.
They must also decide whether invitees should learn about the attendance of 
other invitees or not.
Preferably, all this should be possible in a privacy-preserving fashion.

Realizing this standard feature of \acp{OSN} in a decentralized manner is not 
trivial, because there is no trusted third party which both organizers and 
invitees can rely on.
Since they all depend on only themselves, an implementation of this feature 
must provide security properties that guarantee fairness to all parties 
involved, e.g.\  a protester can verify that the invitation she received was 
actually sent by the organizers.
Moreover, the implementation should also provide privacy settings to protect 
personal information such as the identities of the participants, e.g.\ the 
organizers can restrict to only the invited participants to learn how many 
others have been invited, and only after a protester has agreed and committed 
formally to attend the event, she can learn the identities of other invited 
protesters.

The challenge of implementing this feature without a trusted third party 
becomes greater when the organizers want to allow different types of 
information about the event to be shared with different groups of protesters in 
a secure way because any participant should be able to verify the results to 
detect any possible cheating.
For example, a neutral trusted broker, such as the organizers, could keep 
certain information secret, such as the identities of the invited protesters, 
and only disclose it to those ones who commit to attend the protest. 

In the scheme by \citet{EventsInvitations}, they describe and formalize the 
security and privacy properties outlined above.
More specifically, that the organizer is able to configure who can learn the 
identities of the invited or attending participants, or a more restrictive 
version where only the number of invitees or attendees are revealed.
There is also an attendee-only property that guarantees exclusive access to 
some data, e.g.\ the location of the protest.
These properties are accomplished using several simple primitives:
storage location indirection, controlled ciphertext inference and 
a commit-disclose protocol.
Storage location indirection allows the organizer to control not only who can 
read some the contents of some encrypted data, but also who can access the 
ciphertext.
% XXX Why is this interesting?
Controlled ciphertext inference can be used by the organizer to allow for 
controlled information leaks.
This is needed to achieve properties such as revealing the number of invitees 
but keeping their identities secret.
Finally, the commit-disclose protocol can make some secret available for only 
those participants who committed to attend the protest while, at the same time, 
detect any misbehaving party.
