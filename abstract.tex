% 70--100 words
% Where does that restriction come from? The editors
% Is it still true? I suppose so
% write more fluidly; done
%what problem: privacy-preserving support for protests missing
% why: current tech either not pp or not online
% why care: lack of pp can endanger people, offline alternatives less
% efficient and more difficult
% approach: discuss, develop, adapt PETs for protests
% findings: more out there than used, need input/dialog with activists
% and CS people

While current technologies, such as online social networks, could
facilitate coordination and communication for protest organization,
they can endanger political activists when the control over their data
is ceded to third parties. For technology to be useful for activism,
it needs to be trustworthy and protect the users' privacy; only then
can it be viewed as a potential improvement over more traditional,
offline methods. Here, we discuss a selection of such
privacy-enhancing technologies from a Computer Science perspective in
an effort to open a dialog and elicit input from other perspectives.

% We give an overview of some possible future technologies that can be used for 
% protests.
% We focus on privacy-preserving tools that can be used before, under and after 
% a protest:
% \begin{itemize}
%   \item before, to organize a public protest;
%   \item during, for the organizers and participants to communicate, within 
%     the group or to the outside world;
%   \item after, for organizers to follow up an event by verifying the participation 
%     and computing verifiable statistics, e.g.\ how many participants and in 
%     what area.
% \end{itemize}

\keywords{%
  protest;
  demonstration;
  \Aclp{OSN};
  \Aclp{PET}
}
