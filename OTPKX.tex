\citet{OTPKX} argue that if the adversary controls the entire network, then the
approach to deniability taken by \ac{OTR} and Signal does not suffice.
The problem is that the adversary can record a transcript of all communications
that have taken place.
We know that the \ac{NSA} did exactly that~\cite{XKeyscore}, and specifically 
saved all ciphertexts for later when the decryption key might be 
available~\cite{NSAsavesCiphertexts}.
In this setting it does not matter if anyone can generate a false transcript of
a conversation between Alice and Bob, the regime knows exactly what Alice has 
sent and Bob received and vice versa.
There are more than one way to approach this problem.

The first approach would be to use an anonymizing service, such as 
Tor~\cite{Tor}.
However, for all low-latency solutions, when the entry point and exit from the 
anonymizing network are both controlled by the adversary, then the adversary 
can perform a correlation attack and essentially render the anonymization 
service useless~\cite{AnonymousCommunicationSystems}.
This is in fact the case if the regime controls the nation-wide network while 
critics of the regime, all located in the country, want to communicate in 
real-time.
To make this attack more difficult for the regime's surveillance agency we must 
introduce random delays in our communication.

The second approach would be to ensure deniability even against this strong 
adversary.
This would not hide who communicates with whom, as in our first approach, but 
it provides deniability for the conversations.
The scheme suggested by \citet{OTPKX} makes use of one practical instance of 
deniable encryption~\cite{DeniableEncryption}.
They construct a scheme where Alice and Bob can create \enquote{false 
witnesses} for their conversation.
Basically Alice can create a decryption key such that when used to decrypt the 
ciphertext recorded on the network it will decrypt to a plaintext of Alice's 
choice.
This way she can \enquote{prove} her innocense.
However, the question whether anyone would accept such a \enquote{proof}, 
knowing it can equally well be false, remains open.
