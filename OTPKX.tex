We will now focus on the communication.
Specifically we will focus on communication between pairs of people, e.g.\ 
Alice talking to Bob.
\citeauthor{otr2004} designed a secure protocol for two people communication, 
the \ac{OTR} protocol.
They desired an electronic equivalent of face-to-face conversations, i.e.\ that 
they leave no proofs of any kind behind:
if Alice and Bob has had a conversation, Bob cannot go to Eve afterwards and 
prove anything about what Alice has said --- the same as in a face-to-face 
conversation.
This property is not true for email or when a centralized communications 
service is used.

The standard email system does not provide any security.
A suitable analogy would be that each message is a postcard, i.e.\ it has no 
envelope, so the content and address are visible on it.
This means that the postman can read the cards' contents, their recipients' and
senders' addresses.
(Yes, unlike real postcards these also include the sender's address.)
Furthermore, most postmen use transparent sacks to carry the postcards, so 
everyone along the way can also read the sender's and recipient's address and 
the contents.
However, some postmen have started using non-transparent sacks, i.e.\ encrypted 
connections between the servers, so those postcards can only be read by the 
staff in the post-office.
Thus the email system provides no confidentiality: each email server can read 
the messages, each network operator along the transport route can also read 
(and make a copy of) each email.
However, it is actually worse than that, because the email system provides no 
integrity either.
This mean that the postman, or anyone along the way, can do arbitrary 
modifications to the messages without anyone noticing the difference.
We can safely say that we cannot rely on the email system for neither security 
nor privacy when planning a protest.

When using a centralized communications service, such as Facebook, the level of
security and privacy we can achieve is that the postman carries non-transparent
sacks.
The business model of most such services is to read peoples postcards to better
profile their interests and thus deliver better suiting advertising.
This hides from third parties who is communicating with whom, but this 
information is available internally to the service.
This means that there are ways of learning this, through PRISM~\cite{Prism} for 
example.

Secure email works by employing cryptography: we encrypt the contents of the 
postcard, providing confidentiality, and then add a digital signature to 
prevent modifications.
Thus the recipient is the only one who can read the message and the recipient 
can also verify that the message has not been modified along the way.
To make key management easy, most schemes use public-key cryptography.
This means that we have two keys, one which is public and another which is kept
private.
For encryption, the public key can transform a message to a ciphertext, i.e.\ 
a random-looking text string.
The private key can be used to transform the ciphertext back to the message.
Given only the public key, it is \enquote{impossible} to find the private key.
For signatures, we can use the private key to compute a signature of a message 
and then send the message and its signature.
The recipient can then use the public key to verify the signature of the 
message.
This signature depends on the entire message, so it is impossible to move 
a signature to another message --- as is possible with signatures on paper.
And since it is impossible to find the private key given only the public key, 
no one can create fake signatures.

The problem with the approach to secure email is that the digital signatures 
used provides a property called non-repudiation.
Say that Alice securely sent an email to Bob, if Eve would compromise Bob's 
private key, as many government agencies can, then she would learn that Alice 
--- and no one else --- has sent that message to Bob.
Bob might even give the message and his key to Eve voluntarily or under threat.
This is exactly the property that \citeauthor{otr2004} wanted to remove with 
\ac{OTR}.

\citet{OTPKX} \dots
