\subsubsection{When the Adversary Controls the Network}
\label{WhenAdversaryControlsNetwork}

\textcite{OTPKX} argue that if the adversary controls the entire network, then 
the approach to deniability taken by \ac{OTR} and Signal does not suffice.
The problem is that Eve can record a transcript of all communications
that have taken place.
We know that the \ac{NSA} did exactly that~\cite{XKeyscore} --- and more 
specifically, saved ciphertexts for later when the decryption key might be 
available.
In this setting it does not matter if anyone can generate a false transcript of 
a conversation between Alice and Bob, because Eve knows exactly what Alice has 
sent, what Bob has received and vice versa.
The argument of this class of protocol is that Alice and Bob have the 
possibility to deny anything about the conversation since it cannot be 
decrypted.
This seems extra problematic when even the free countries in the world suggest 
that there must be ways to break this 
encryption~\cite{BackDoorEncryption}\footnote{%
  We refer the reader to the text by \textcite{KeysUnderDoormats} for further 
  reasons for why this is a bad idea.
}.

There are more than one way to approach this problem.
The first approach would be to use an anonymizing service, such as 
Tor~\cite{Tor}.
This way, Eve would not know that Alice communicates with Bob, only that
Alice communicates with someone.
However, Alice and Bob are located in the same country and Eve controls the 
nationwide network.
For all low-latency anonymizing networks (such as Tor) where the entry point 
and exit are controlled by Eve, Eve can perform a time-correlation 
attack\footnote{%
  This means that Eve records the time of when each message enters the network 
  (entry distribution) and the time when each message exits the network (exit 
  distribution).
  Due to the low-latency property, these distributions will be related and Eve 
  can infer to whom Alice sent her message.
} and essentially render the anonymization service 
useless~\cite{SystemsForAnonymousCommunication}.
To make this attack more difficult for Eve, the system must introduce random 
delays in our communication\footnote{%
  The delays must transform the exit distribution to a distribution more 
  similar to the uniform distribution, then Eve's statistical analysis will 
  become more difficult.
}.
(We will return to this topic in \cref{MessageDistribution}.)
But despite all this, Eve can still ask Alice to decrypt the conversations, 
either she complies or claims that she does not know the key.

The second approach would be to ensure deniability even against this strong 
adversary.
This would not hide who communicates with whom, as in our first approach, but 
it provides deniability for the conversations.
The scheme suggested by \textcite{OTPKX} makes use of one practical instance of 
deniable encryption~\cite{DeniableEncryption}.
They construct a scheme where Alice and Bob can create \enquote{false proofs} 
for their conversation.
In essence, Eve records all traffic.
When she approaches Alice and asks her to provide a key to decrypt the recorded 
traffic, Alice can create a decryption key such that when Eve decrypts the 
recorded traffic will receive a plaintext of Alice's choice.
This way Alice can plausibly deny any allegations.
However, the question whether Eve would actually accept such a \enquote{proof}, 
knowing it might equally well be false, remains open.
