The problem of authentically associating data with a physical event is 
difficult.
We can essentially divide it into the following requirements:
\begin{requirements}
  \item\label{CreatedBeforeEnd} Prove that the data was created before the end 
    of the event.
  \item\label{CreatedAfterStart} Prove that the data was created after the 
    start of the event.
  \item\label{SpatiallyRelated} Prove that the data is spatially related to 
    the physical location of the event.
\end{requirements}
\Cref{CreatedBeforeEnd,CreatedAfterStart} together bind the data to the time of 
the event whereas \cref{SpatiallyRelated} binds the data spatially to the 
event.

Now, consider the scenario of Alice taking a photo during the demonstration and 
posting it online.
What can we say about this photo?
First, we can say that it was created before we viewed it, so 
\cref{CreatedBeforeEnd} is fulfilled if we view it in relation to the event.
If it was submitted to a service that we trust, then we can also trust the 
time-stamp of the service.
Furthermore, we can also consider it spatially related to the physical location
(\cref{SpatiallyRelated}) if we can convince ourselves that the photo is 
depicting the physical location and not any kind of \enquote{reconstruction}, 
e.g.\ it is computer generated or a photo of a similarly looking location.

\Cref{CreatedAfterStart} is more difficult to achieve.
In the above scenario, there is nothing that prevents Alice from submitting an 
older photo which is spatially related to the event.
In the security field \cref{CreatedAfterStart} is captured by a property called 
\emph{freshness}.
The freshness property is usually achieved by requiring that the data depend on
an unpredictable value.
The unpredictable value is commonly a value chosen randomly by the verifier, 
but there are ways to do this without the verifier too --- the basic 
requirement is that it is not under the prover's control, i.e.\ Alice in this 
scenario.
The idea is that we require Alice to include the front-page of a particular 
newspaper in the photo, since the exact front-page is difficult for Alice to 
predict in advance.
This would of course not work in practice, since Alice can manipulate the photo 
afterwards to superimpose the required newspaper frontpage, but it illustrates 
the idea.
There is, however, a subfield of digital forensics that work with image 
manipulation detection, so there are methods that would prevent at least 
Alice's easiest manipulation attempts.

We can see that the techniques to achieve 
\cref{CreatedBeforeEnd,CreatedAfterStart,SpatiallyRelated} depends on the type 
of data.
We used as an example above a photo, next we will look at another type of data.
