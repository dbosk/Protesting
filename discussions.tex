So far we have treated only one-to-one conversations, i.e.\ Alice and Bob 
talking to each other.
However, there are usually more than two people organizing a protest, and so we 
need to hold discussions with more than only two people at a time.
In this situation there are two approaches to solving the communication:
simultaneous pairwise (one-to-one) communication between all participants or 
multicast communication.
Furthermore, how the messages are distributed is also important, because Eve 
might be able to learn who the participants are.

\subsubsection{Group Communication Properties}
\label{GroupProperties}

When a group uses pairwise communication, every member of the group will set up 
a one-to-one channel to each other member of the group.
(Illustrated in \cref{fig:FullMeshNetwork}).
\begin{figure}
  \centering
  \includegraphics[width=0.3\textwidth]{FullMeshNetwork.eps}
  \caption{%
    A full-mesh network with six nodes.
  }
\end{figure}
Each pair-wise channel is as described above, in \cref{Communicating}.
Then for every message Alice wants to send to the group she has to send it to 
every participant.
This would allow Alice to cheat, e.g.\ she can send \enquote{Who wants to 
  overthrow the regime?} to everyone except to Bob, to whom she instead sends 
\enquote{Who wants to order pizza?}.
This opens up for the Byzantine Generals' problem~\cite{ByzantineGenerals}, 
where malicious actors can lie to honest actors to disrupt operation.
\citet{ByzantineGenerals} in fact proved that it is impossible for the honest 
parties to recover and identify the malicious parties if the malicious parties 
exceed a third of the participants.
  
Although Alice's ability to say different things to different participants is 
in itself a desirable property from Alice's perspective --- she would like to 
lie to suspected regime agents --- this property can at the same time be 
undesirable due to the Byzantine Generals' problem.
For this reason group communication must provide better properties, namely that 
everyone hears who said what and when, thus forcing Alice to say the same thing 
to all participants.
In such a scheme, when Bob replies \enquote{I do, shall we say tonight?} the 
others will see that Bob is replying to something they did not see and not to 
the question \enquote{Who wants to overthrow the regime?}.

\citet{multiotr2009} tried to extend the \ac{OTR} protocol to a multi-party 
setting.
This did not result in a concrete protocol implementation, and the resulting 
protocol they suggested was also very complex.
It also had some undesirable limitations, for instance, the scenario that Bob 
receives a question which is different from everyone else's is only detected at 
the very end of the conversation.
As is pointed out by \citet{TSgroups}, asynchronous communication today has no 
real end, which makes the approach of \citet{multiotr2009} even less appealing.
Due to this, \citet{SignalApp} implements group chats as simple pair-wise 
conversations.
With additional meta-data they can ensure consistent history, however, this is 
not yet implemented~\cite{TSgroups}.
A technique that could be used for this is to include a message digest of the 
entire conversation history with each message.
A message digest is computed using a one-way function, i.e.\ its output is 
unpredictable and its input is impossible to compute given only the
output. %TODO: explain more clearly
This means that the message digest included in Bob's reply and the one computed 
by the other participants above would differ, thus everyone learns that the 
conversation history is inconsistent and should no longer be trusted.
Due to the unpredictable property of the one-way function, Alice cannot phrase 
the two different messages in such a way that they yield the same message 
digest in the history either.
But despite this, the other participants cannot determine if it is Alice or Bob
who is lying about the message history --- Alice could send the same message to
everyone and still Bob could try to frame her.

\subsubsection{Message Distribution}
\label{MessageDistribution}

\textcite{PPACinPubFS} analysed two dichotomous models of communication: the 
pull model and the push model.
Alice wants to send a message to Bob and Carol.
In the pull model, Alice would publish her message in a place known to Bob and 
Carol.
Bob and Carol visits this place periodically to check if Alice has published 
any new messages.
(If there is a new message, they make a copy for themselves.)
In the push model, Alice drops her message in Bob's and Carol's letter boxes.
(One copy for each.)
Email and text-messaging are best modelled using the push model (see 
\cref{GroupCommunication}).
For both models, Eve can analyse the communication patterns to infer (a part 
of) the social graph unless Alice, Bob and Carol use some countermeasures.

%\citeauthor{PPACinPubFS} found that achieving privacy in the pull model is 
%technically easier than in the push model.
%In fact, achieving strong privacy in the push model is very
%difficult. %TODO: explain why (move to end)

\paragraph{The Pull Model}

We can start by looking at the pull model for communication.
Alice wants to distribute a message to Bob and Carol, the participants in 
a discussion.
In the pull model they actively ask Alice (or an intermediary) for new messages 
at regular intervals.
To form a protocol around this model, Alice, Bob and Carol can agree on 
a location where Alice puts her messages\footnote{%
  Technically, this can be implemented in a similar fashion as the \ac{DHT}, as 
  mentioned in \cref{UserSearch}.
  This would make it more difficult for Eve to censor it compared to the 
  centralized systems.
}.
When Alice wants to send a new message, she writes it to this particular 
location.
When Bob and Carol want to, they can read from the location to see if there are 
any new messages.

Suppose that Eve controls the network that Alice uses\footnote{%
  This is reasonable considering that we saw earlier that the \ac{NSA} and 
  \ac{GCHQ} achieve similar results in free countries.
}.
Since we have a decentralized system in mind, we can also assume that anyone 
(especially Eve) can read anything from the storage.
%This is why Alice encrypts all her messages for the recipients' keys.
%Also, Alice does not want to be associated with the message, not authorship nor 
%posting it.

The first thing we can say about this situation is that Alice would like to 
have confidentiality for the messages' contents, so that Eve cannot read her 
messages.
Alice would also like to have integrity for her messages, so that Bob and Carol 
can be sure that they are from Alice and that Eve has not modified them.
Many systems provide these two properties, e.g.\ \ac{PGP} does this for email 
(and could be applied here as well).
However, Alice also wants to hide the sender and recipients, which many systems 
(including \ac{PGP}) do not provide.
There is a class of encryption schemes called \ac{ANOBE} schemes.
This type of scheme provides confidentiality while hiding the sender and the 
intended recipients.
If Alice can write the message anonymously to the storage and the message is 
encrypted using \iac{ANOBE} scheme, then it will be difficult to determine the 
sender.
Furthermore, if the recipients fetch the messages anonymously too, then the 
recipients are also hidden.
The idea is as follows: if Eve cannot distinguish between Bob and Carol 
fetching a message, then it might just as well only be Bob who fetches messages 
from this location --- Eve cannot tell the difference.
(We will return to this problem later.)

The problem of integrity remains.
There are two approaches: digital signatures and \acp{MAC}.
If Alice, Bob and Carol agree on a commonly shared \ac{MAC} key, then they can 
use \acp{MAC} to ensure integrity.
One advantage of \acp{MAC} is that anyone who can verify the authenticity of 
\iac{MAC} can also create one (as was pointed out above).
With digital signatures on the other hand, if Alice signs a message it is clear 
that Alice is the only one who could have signed it.
(It is important that only Bob and Carol know that Alice owns the private key, 
and that it remains anonymous to Eve.)
But this provides Eve with something to track messages by, all messages signed 
by the same key are related.
With \acp{MAC}, Bob and Carol could also have authored the message and Eve 
cannot determine which messages are related either.
This means that for a discussion, any of the participants would be equally 
likely to be the author of a given message.
However, this relies on the anonymity of the actors.

\paragraph{The Push Model}

In the push model, Bob and Carol have one location each where Alice will drop 
her messages.
(She can achieve confidentiality and integrity similarly as in the pull model.)
One thing we can observe is that the recipients are not as hidden as in the 
pull model, even if we assume anonymity.
Eve can observe that someone (Alice, but Eve does not know that) puts two 
messages at the same time.
Eve can then observe these locations to see when someone (Bob or Carol, but Eve
does not know that either) reads messages from those locations.
The main problem with the push model is that it reveals more meta-information 
than the pull model does.
With the push model Eve can build herself a map of the social graph\footnote{%
  Since Eve works with probability distributions, this would be an 
  approximation of the social graph.
  But her approximation can come very close to the real one.
}.
Then she only needs to map the real identities of Alice, Bob and Carol to these 
anonymous identities.

\paragraph{Privacy}

%Say that Alice, Bob and Carol have one inbox each, similarly as in the email 
%system or Signal.
%Eve monitors the network on a national level.
%Now Eve can see one message originating from Alice, going to a server beyond 
%Eve's reach, and soon two equally-sized messages return from the server 
%near-simultaneously to Bob and Carol\footnote{%
%  Or equivalently, Eve observes where these messages end up in the storage and 
%  later observes Bob and Carol fetching these messages.
%}.
%(As we pointed out above, this is what is called a time-correlation attack.)
%This is what happens when Alice, Bob and Carol do not have perfect anonymity, 
%e.g.\ by using Tor or simply using Signal without any anonymizing service.
%Eve can relate Alice, Bob and Carol to each other.
%Now let us try to make this more difficult for Eve.
%
Say that Alice, Bob and Carol can access the storage system for messages 
anonymously, i.e.\ Eve can only observe when Alice, Bob and Carol does 
something --- but not what they do --- and when something happens in the 
storage system --- but not who does it.
Despite this anonymity, Eve can still do a correlation attack.
For example, Eve can temporarily detain Bob (or turn off his network 
connection) and observe the change in the distribution of reads from the 
storage.
In the push model, she will observe that someone stopped reading from one of 
the locations, i.e.\ Bob's location.
The same argument can be applied to the pull model case: when Eve detains Bob, 
she can observe a change in the probability distribution of reads from where 
Alice puts her messages.
In fact, even if several people share locations, this simply slow Eve down.

As was pointed out above (\cref{WhenAdversaryControlsNetwork}), the solution to 
this type of attack is to add noise to make these changes in distribution 
indistinguishable.
\citeauthor{PPACinPubFS} suggested that differential privacy\footnote{%
  Differential privacy guarantees that if we remove any \emph{one} data item, 
  the distribution will change below a given threshold.
} will probably be the best trade-off between privacy and efficiency.
In fact, parallel to the work of \citeauthor{PPACinPubFS}, 
\textcite{Vuvuzela,Alpenhorn} designed a protocol for one-to-one communication 
based on differential privacy.
There are still some limitations, e.g.\ everyone must be online and 
participating in the protocol all the time --- 24 hours every day.

%If one of the participants makes a mistake, then the regime's agents will have 
%a starting point to target.
%For example if a participant uses the same inbox for communication with all his 
%friends, and not only the participants in the plot against the regime, then one 
%of his other contacts might not be as concerned with staying anonymous.
%The consequence is that the regime can see the identity of someone sending 
%messages to an inbox of their interest.
%Then they can target this person and learn which friend owns the inbox of 
%interest.
%Then they can proceed to targeting one of the protest organizers.
%This type of attack will not work when the communication is according to the 
%pull model, since there the agency must attack each anonymous connection.
%%TODO: reviewer: explain what you mean by attack here
%


