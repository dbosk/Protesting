One problem in physical protesting that has not yet been solved satisfactorily 
is the verification of the participation in a protest.
After many protests the counts of demonstrators of police and the organizers 
differ, in some instances the difference can be up to hundreds of 
thousands~\cite{ExampleProtestCount}.

Many of the current methods to count the participants are based on computer 
vision, i.e.\ object recognition through image analysis.
A demonstration is very similar to voting, both are many individuals expressing 
their opinion.
Hence it is desirable to have similar properties for verifying the 
participation in a protest.
Instead of votes we could have participation proofs and the following 
properties:
\begin{properties}
  \item\label{VerifEligibility} Everyone can verify that every participation 
    proof is correct.
  \item\label{VerifIndividual} Every participant can verify that its 
    participation proof is included in the count.
  \item\label{VerifUniversal} Everyone can verify that the count is according 
    to the participation proofs.
\end{properties}
This is difficult to accomplish with computer vision methods.
The properties outlined above are indeed desirable, e.g.\ then the \ac{UN} can 
verify protests happening in a country and the country cannot deny it, thus the 
\ac{UN} can apply pressure if needed.

Additionally, in voting, the cast votes are not linkable to the identity of the 
casters.
Thus it is not a problem to reveal the identites of those who participated in 
the vote, since they could have voted for any alternative.
We would also like to have the corresponding property in verifying the 
protest.
The very nature of a protest is different though, we do not have any choice: if 
we participate we support the cause of the protest.
Consequently we want to verify the participation of a protest without 
identifying individuals who participated.
Otherwise the regime's agents can identity all the participants and simply 
\enquote{make them disappear}.

For real-world protests we need to bind participants to the same physical 
location at a reasonably similar time, i.e.\ within the area and duration of 
the protest.
\citet{PROPS} developed a decentralized \ac{LPS} which provides a participant 
with a verifiable proof of having been at a location at a certain time, 
something we call \iac{LP}.
It is decentralized because there is no central authority that verifies the 
location, instead peers act as witnesses.
Then a third-party can verify the authenticity of the \ac{LP}, by verifying the 
witnesses signatures, and can thus be sure that the person has indeed been in 
the location.

Bosk, Gambs and Buchegger are currently exploring (work in progress) the 
possibility of turning such \iac{LPS} into a system for verifying the 
participation in protests.
The idea is that each participant generates \iac{LP} during the protest, where 
(some of) the other protesters act as witnesses, then the \acp{LP} can be used 
to compute the participation count.

