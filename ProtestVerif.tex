One problem in physical protesting that has not yet been solved satisfactorily 
is the crowd counting problem, or more generally, the verification of the 
participation in a protest.
% XXX I would like a reference for this
After many protests the demonstrator count by police and that by organizers 
differ, in some instances the difference can be hundreds of thousands.
Many of the methods to count the participants in the literature are based on 
computer vision, i.e.\ object recognition through image analysis.
With this type of technique a third party has no way of verifiying that the 
count is correct, hence the dispute between organizers and police over these 
counts.
A demonstration is very similar to voting, both are many individuals expressing 
their opinion.
Hence it is desirable to have similar properties for verifying the 
participation in a protest, where this verification step is at the core.

In the context of voting protocols, in particular e-voting protocols, there are 
three desirable properties for verification:
\begin{properties}
\item\label{UniversalVerif} Anyone can verify that the result is according to 
  the cast votes.
\item\label{EligibilityVerif} Anyone can verify that each vote cast is 
  legitimate.
\item\label{IndividualVerif} Every voter can verify that its vote is included 
  in the result.
\end{properties}
We can transfer these properties to the case of protest participation, then 
each vote would be replaced by a proof of participation.
\Cref{EligibilityVerif} would in this case mean that anyone can verify that 
each participation proof belongs to a unique individual, i.e.\ to prevent any 
Sybil attack (\cref{SybilAttacks}).

The three verifiability properties above are indeed desirable, e.g.\ then the 
\ac{UN} can verify protests happening in a country and the country cannot deny 
it, thus the \ac{UN} can apply pressure if needed.
However, the properties are difficult to accomplish with computer vision 
methods, especially if nobody from the \ac{UN} has been on location to collect 
the image material for the analysis --- since then the authenticity can be 
questioned as outlined in \cref{DataAuthenticity}.

Additionally, in voting, the cast votes are not linkable to the identity of the 
casters.
Thus it is not a problem to reveal the identites of those who participated in 
the vote, since they could have voted for any alternative.
We would also like to have the corresponding property when verifying the 
protest.
The very nature of a protest is different though, we do not have any choice: if 
we participate we support the cause of the protest. %TODO: what about
                                %counterprotests in the same location
Consequently we want to verify the participation of a protest without 
identifying individuals who participated.
Otherwise the regime's agents can identity all the participants and simply 
\enquote{make them disappear}. %TODO: reviewer: unclear

Finally, for real-world protests we need to bind participants to the same 
physical location at a reasonably similar time, i.e.\ within the area and 
duration of the protest.
This means that for a system like this to work, we also need the requirements 
from \cref{DataAuthenticity}: proof that the data was created after the start 
of the protest, proof that it was created before the end, and proof of spatial 
relation to the location.
For the last property, \citet{PROPS} developed a decentralized \ac{LPS} which 
provides a participant with a verifiable proof of having been at a location at 
a certain time, something we call \iac{LP}.
It is decentralized because there is no central authority that vouches for the 
location, instead peers act as witnesses.
Then a third-party can verify the authenticity of the \ac{LP}, by verifying the 
witnesses signatures, and can thus be sure that the person has indeed been in 
the location.
Bosk, Gambs and Buchegger are currently exploring (work in progress) the 
possibility of combining such \iac{LPS} with the verifiability properties for 
voting protocols into a system for verifying the participation in protests.
The overall idea is that each participant generates \iac{LP} during the 
protest, where (some of) the other protesters act as witnesses, then the 
\acp{LP} can be used to compute the participation count with all the 
verifiability properties above.
%TODO: reviewer: what about cellphones to count attendance
