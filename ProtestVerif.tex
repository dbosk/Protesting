In general, protesting is very similar to voting: both are many individuals 
expressing their opinion.
Hence we desire to have similar properties of verification and privacy for 
participation in a protest as there is for voting.
As was pointed out in the introduction, an online protest can be modelled as 
a petition, which in turn can use electronic voting protocols.
We are interested in translating these properties into a demonstration, i.e.\ 
a street-protest.

In the context of (electronic) voting protocols, there are three requirements 
for verification~\cite{VerifyingPrivacyPropertiesOfVotingProtocols}:
\begin{requirements}[V]
\item\label{EligibilityVerif} Eligibility: anyone can verify that each vote 
  cast is legitimate.
\item\label{UniversalVerif} Universal verifiability: anyone can verify that the 
  result is according to the cast votes.
\item\label{IndividualVerif} Individual verifiability: every voter can verify 
  that their vote is included in the result.
\end{requirements}
We can translate these to the case of participation in a demonstration, then 
each vote would be replaced by a proof of participation (in either the protest 
or any counter-protest).
\Cref{EligibilityVerif} would in this case mean that anyone can verify that 
each participation proof belongs to a unique individual, i.e.\ to prevent any 
Sybil attack (as discussed in \cref{SybilAttacks}).

The three verifiability properties above are indeed desirable, e.g.\ then the 
\ac{UN} can verify protests happening in a country and the country cannot deny 
it, thus the \ac{UN} can apply pressure if needed.
However, these properties are difficult to accomplish with computer vision 
methods, especially if nobody that the \ac{UN} can trust has been on location 
to collect the image material for the analysis --- otherwise the authenticity 
can be questioned (as discussed in \cref{DataAuthenticity}).

We also need privacy in addition to the verification requirements.
In voting, we have the following requirements:
\begin{requirements}[P]
\item\label{VotePrivacy} Vote privacy: the vote does not reveal any individual 
  vote.
\item\label{ReceiptFreeness} Receipt freeness: the voting system does not 
  provide any data that can be used as a proof of how the voter voted.
\item\label{CoercionResistance} Coercion resistance: a voter cannot cooperate 
  with a coercer to prove the vote was cast in any particular way.
\end{requirements}
\Textcite{VerifyingPrivacyPropertiesOfVotingProtocols} showed that 
\cref{CoercionResistance} implies \cref{ReceiptFreeness}, which in turn implies
\cref{VotePrivacy}.
We can see that \cref{VotePrivacy,ReceiptFreeness} are desirable in Alice's 
situation: there should not be any proof which binds Alice to the participation 
of the demonstration (\cref{ReceiptFreeness}), because she does not want to 
explicitly reveal her opinions to the government (\cref{VotePrivacy}) due to 
the risk of reprimands.
\Cref{ReceiptFreeness} implies that even if Alice is arrested, the data should 
not provide any proof to the regime's agents that can reveal Alice's opinion.

A demonstration is very different from voting in one sense: Alice must be 
physically present and that very precense shows her support for the cause.
In voting, on the other hand, Alice has multiple options which are not revealed
by her mere precense.
As we indirectly pointed out earlier, we focus on the privacy provided to Alice 
and Bob by the data.
So as long as Alice and Bob can conceal their identities at the demonstration 
and escape without arrest, their support is recorded in the data while their 
privacy is not violated.
(Following this line of thinking, it can actually be beneficial for the privacy of 
the demonstrators to mix with the participants from any counter-demonstrations 
--- as long as the correct counts for each demonstration can still be ensured, 
which is the idea.)

The main problem is how to ensure the connection between the physical location 
and the data used for verification, i.e.\ fulfil all the requirements for 
authenticity, verification and privacy given above.
We must bind participants to the same physical location at a reasonably similar 
time, i.e.\ within the area and duration of the demonstration.
In the case of the Korean demonstrations~\cite{2016DemonstrationsInSeoul}, this 
was in one place during an entire day and then repeated for several 
weekends.
In the case of the Women's Marches in the US~\cite{2017WomensMarchesInUS}, they
were in several locations at the same time.

%This means that for a system like this to work, we also need the requirements 
%from \cref{DataAuthenticity}: proof that the data was created after the start 
%of the protest, proof that it was created before the end, and proof of spatial 
%relation to the location.

\textcite{PROPS} developed a decentralized \ac{LPS} which provides 
a participant with a verifiable proof of having been at a location at a certain 
time, something we call \iac{LP}.
It is decentralized because there is no central authority that vouches for the 
location, instead peers act as witnesses.
Then a third-party can verify the authenticity of the \ac{LP}, by verifying the 
witnesses' signatures, and can thus be sure that the person has indeed been in 
the location.
Bosk, Gambs and Buchegger are currently exploring (work in progress) the 
possibility of combining such \iac{LPS} with the verifiability and privacy 
requirements discussed above.
The overall idea is that each participant generates \iac{LP} during the 
demonstration, where (some of) the other protesters act as witnesses, then the 
\acp{LP} can be used to compute the participation count with all the 
authenticity, verifiability and privacy requirements above.
%TODO: reviewer: what about cellphones to count attendance
% This is actually done in \cite{2016DemonstrationsInSeoul}, it requires some 
% assumptions --- but they use wifi signals instead of the mobile network.
% In either case, we pointed out that one wants the phone in flight mode due to
% surveillance.
