\section{During a Protest}
\label{DuringProtest}

During a protest, organizers and demonstrators might want to communicate, 
either among themselves or to the outside world.
There are a few problems with communication during a protest.
If the participants use the phone network, which generally is controlled by the 
government, they can be tracked and bound to the location by their phone.
If they communicate over the phone network they can still use the techniques 
outlined in \cref{BeforeProtest}.
If they don't want to be tracked, they must use another network infrastructure 
that is not controlled by the government.

The communication to the outside world can have at least two purposes.
The first one is simply to try to get more people to come to the demonstration.
The second is when a demonstrator wants to store something for posterity.
This can be a photo capturing police brutality or a part of 
a proof-of-demonstration (explained in \cref{AfterProtest}).
\citet{PPACforPubFS} presents work done in the area of privacy-preserving 
access control in distributed storage systems.
This is important since it outlines some possibilities and limits for such 
systems.
The problem in this scenario is that we don't want to be identifiable as 
a demonstrator, as this might yield repercussions.
Thus we can to share the data anonymously, no one can tell with whom we've 
shared what or if they've read it.
However, we still want to verify authenticity as it might otherwise be the 
regime spreading disinformation.

In this case it is also not straight forward to just apply the techniques in 
\cite{OTPKX} to achieve deniability.

