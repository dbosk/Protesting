\subsection{Searching for Your Friends}
\label{UserSearch}

For Alice and Bob to be able to communicate, they must have a way to set-up 
their secure communication.
The particular instance we will cover here is how Alice can find Bob's profile 
in \iac{DOSN}, yet prevent the regime from also finding it.

Let us assume that Alice knows more about Bob than the regime does.
Then a knowledge threshold can be enforced using cryptographic techniques, 
i.e.\ to guarantee that Bob can only be found if the party searching for him 
can present enough details about him (\enquote{find me if you know enough about
me}).

\Textcite{ThresholdUserSearch} presents two protocols that have different 
advantages and disadvantages.
Neither of them relies on any central repository of user data.
This avoids the biggest risk to user data: the leakage of a central database 
with sensitive information about a large number of people.
Practically, the protocols can be implemented in a completely decentralized way 
using \iac{DHT}, which is a standard component of
decentralized systems to store, locate, and retrieve data.

The proposed protocols allow Alice and Bob to register their identifiers, e.g.\ 
links to their profile pages, e-mail addresses or other contact information.
They can also specify the required knowledge that is needed to find this data,  
e.g.\ name, city, workplace and date of birth.
One protocol guarantees this knowledge-threshold by encoding the storage 
location of Bob's identifier using the required knowledge attributes.
Only users that know these attributes can construct a valid lookup request for 
the \ac{DHT} that will return Bob's identifier.
The other protocol stores Bob's identifier encrypted in the \ac{DHT} and uses 
threshold secret-sharing techniques\footnote{%
  This is a cryptographic technique where a secret value is split in \(n\) 
  shares.
  Without at least \(t\leq n\) of these shares, the secret cannot be 
  reconstructed.
} to guarantee that no user with less than the required number of attributes 
can decrypt a stored identifier.

Neither protocol can provide perfect protection.
In the worst-case of a targeted attack, an adversary with profound background 
knowledge about Bob will likely succeed.
For example, we cannot protect Bob's identifier if the adversary knows as many 
attributes about him as Alice does.
At the same time, both protocols protect Bob fairly well from large-scale 
crawling attacks, as the search space of all possible attribute combinations is 
too large and the protocols transform the registered user data in such a way 
that inferences from the publicly stored data are infeasible.
Even if the adversary focuses her effort to only crawl the data of a specified 
subset of the user-base, e.g.\ all persons working at a specific organization, 
the proposed protocols offer good protection. 

The knowledge-threshold is an individual user parameter, so users that
consider themselves to be more exposed to risks can choose a higher
knowledge-threshold to increase their protection at the cost of lower
usability --- as a higher threshold makes it harder for other legitimate
users to find them.
In that sense, the presented protocols allow users to individually balance 
their findability and privacy requirements.

