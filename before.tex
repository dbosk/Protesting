\section{Before a Protest}
\label{BeforeProtest}

There are several issues related to protesting in the stage before.
First comes the initial discussion between potential organizers, second comes 
the problem of scheduling this with participants.
For the initial discussion, the potential organizers don't want the regime's 
intelligence services to identify them as such.
One of the most popular secure-messaging protocols is \ac{OTR}, lately 
popularized in smartphones through Signal (formerly TextSecure).
\citet{OTPKX} proposed a scheme with the same properties as \ac{OTR} but with 
more deniability.
Another problem in this stage is to find other's user profiles.
\citet{ThresholdUserSearch} designed a scheme for targeted user search by means 
of user defined knowledge threshold.
Protesters must be able to find each other in the networks, but we don't want 
the oppressive regime to do the same.

For the scheduling of a protest, there are in turn several problems that must 
be addressed.
From the organizer Alice's perspective, she wants to protect herself from being 
arrested for organizing a protest.
So Alice needs to protect herself from the possible participants, as one of 
them can be agent Eve of the intelligence services of the regime.
From the participant Bob's perspective, he wants to protect himself from being 
arrested for committing to participate in a protest.
So Bob needs to protect himself from the organizer and the other participants, 
as any of them can be Eve.

When organizing a protest, what Alice and Bob want to agree on is a time, 
a place and to ensure enough people will show up at that time and place.
Alice and Bob also wants Eve to learn as little as possible of the plans so 
that she cannot curtail the protest.

\citet{EventsInvitations} presented a distributed protocol without the need of 
\iac{TTP}.
This protocols allows for different privacy settings:
\begin{itemize}
\item Alice discloses nothing to Bob, except the time and the place;
\item Alice discloses everything --- who the invitees are, who of those have 
  already committed etc.;
\item and every combination of settings in between.
\end{itemize}
Further, if Alice doesn't keep her promises, Bob has a proof which he can 
publish to everyone to show that Alice cheated.
Likewise, if Bob commits to attending, Alice has proof that Bob has done so and 
can show to everyone that Bob isn't present although he said he would.

One topic that must be explored is to adapt this protocol to introduce 
deniability.
Another interesting feature to include would be the choice of location.
With this, all participants can jointly agree on not only a time, but also 
a location.
This would help against the problem of announcing the location in advance.

%% EI contribution :: START
There are some tasks to accomplish prior to the protest itself that the organizers 
of the gathering need to arrange, for example, decide who are the most suitable 
candidates to attend the event, how to let them know about the protest and what 
preliminar information they should get or, later on, learn about the attendance of 
the invited ones in a privacy-preserving fashion.

However, realizing this standard feature of OSNs in a decentralized manner is not trivial 
because there is not a trusted third party to which both organizers and, invited and 
attending protestors can rely on. Because they all depend on themselves, an 
implementation of this feature must provide security properties that guarantee fairness 
to all parties involved, \eg a protestor can verify that the invitation she received 
was actually sent by the organizers. Moreover, the implementation should also provide 
with privacy settings to protect personal information such as the identities of 
the participants, for example, the organizers can restrict to only the invited participants 
to learn how many others have been invited, and only after a protestor has agreed 
and committed formally to attend the event, she can learn the identities of other 
invited protestors.

The challenge of implementing this feature without a trusted third party becomes 
greater when the organizers want to allow different types of information about the 
event to be shared with different groups of protestors in a secure way because any 
participant should be able to verify the results to detect any possible cheating. 
For example, a neutral trusted broker, such as the organizers, could keep certain 
information secret, such as the identities of the invited protestors, and only disclose 
it to those ones who commit to attend the protest. 

In our scheme, we describe and formalize a set of security and privacy properties 
to configure who can learn the identities of the invited or attending participants, 
or a more restrictive version in the count of invitees or attendees, and an attendee-only 
property that guarantees exclusive access to some information, for example, the 
location of the protest. By means of a set of privacy enhancing tools, such as storage 
location indirection, to control not only who can read some the contents of some 
encrypted information but also who can access the ciphertext; or the controlled 
ciphertext inference, to allow for controlled information leaks such as learning 
the number of invited participants but not their identities; and a commit-disclose 
protocol, to make some secret available for only those participants who committed 
to attend the protest while detecting at the same time any misbehaving party; we 
propose a trusted-third party free architecture, together with standard cryptographic 
primitives in our decentralized scenario.

%% EI contribution :: END

