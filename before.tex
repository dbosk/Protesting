\section{Before a Protest}
\label{BeforeProtest}

Alice is an activist in an authoritarian regime and she want to organize 
a demonstration.
This means that the regime wants to stop Alice while she prepares the 
demonstration.
Consequently, Alice wants to do all preparations in such a way that she 
minimizes the risk of interference.
In this section we will focus on two aspects: the communication between the 
activists and their agreement on the details of the demonstration event.

\paragraph{Communication}

Alice and her co-organizers must communicate with each other.
A trivial solution to the communication problem is the traditional face-to-face 
meeting --- with the trade-off that the invited attendants should be able to 
meet at the same time in the same place.
This is not always easy to achieve, so Alice wants to complement this by 
electronic communications.
Thus Alice wants to communicate with Bob by means of a secure channel to 
prevent the regime's agents from eavesdropping on their conversations.

Secure communication can be divided into two problems: bootstrapping and the 
actual communication.
There are several tools that we will discuss related to this.
In \cref{UserSearch}, we will discuss how Alice and Bob can find each other in 
\iac{DOSN}, this relates to the bootstrapping problem.
Then we discuss security and privacy properties of the communication problem, 
we focus on one-to-one secure communications in \cref{Communicating}.
There is also the case where Alice wants to talk to more people than just Bob, 
and this case is discussed in \cref{Discussions}.

\paragraph{Agreement}

Alice and the co-organizers must agree on a time and place to hold the 
demonstration.
This can also be extended to including interested participants.
For example, the organizers might be interested in estimating how many invited 
participants are really committed to attend the event, but in such a way that 
they do not reveal the details --- which the regime can use to thwart it --- 
such as the location and any identities.
At the same time, the participants who have committed to attend may want to 
have assurances that they will be told the details of the protest if they 
express their commitment to the organizers.
We discuss some aspects of this problem in \cref{InvitingParticipants}.


% Searching for your friends
\subsection{Searching for Your Friends}

So the challenge is to protect user data from malicious adversaries but
at the same time making users findable for other legitimate users. To
distinguish between these two cases, it is assumed that legitimate users
possess more information about a target user than the adversary. Then a
knowledge threshold is enforced using cryptographic techniques, to
guarantee that a user can only be found if the party searching for her
can present enough details about her (\enquote{find me if you know enough about
me}). Two protocols implementations are presented that have different
advantages and disadvantages. Both do not rely on any central repository
of user data but are suitable to be implemented in a completely
decentralized way using a Distributed Hash Table (DHT). This avoids the
biggest risk to user data: the leakage of a central database with
sensitive information about a large number of people.
The proposed protocols allow users to register their identifiers (e.g.\ 
links to their profile pages, e-mail addresses or other contact
information) and specify the required knowledge that is needed to find
this information (e.g.\  name, city, workplace and date of birth). One
implementation guarantees this knowledge-threshold by encoding the
storage location of the registered user identifiers using the required
knowledge attributes. Only users that know these attributes can
construct a valid lookup request for the DHT that will return the
desired user identifier. The other protocol stores user identifiers
encrypted in the DHT and uses threshold secret-sharing techniques to
guarantee that no user with less than the required number of attributes
can decrypt a stored identifier.

Both protocols cannot provide perfect protection. In the worst-case of a
targeted attack, an adversary with profound background knowledge about
the target user will likely succeed. For example protecting the user
identifier cannot be accomplished if the adversary knows as many
attributes about the target user as legitimate users do.
At the same time both schemes protect the users fairly well from
large-scale crawling attacks as the search space of all possible
attribute combinations is too large to brute-force and the protocols
transform the registered user data in such a way that inferences from
the publicly stored data are infeasible. Even if the adversary focuses
her effort to only crawl the data of a specified subset of the user-base
(e.g.\  all persons working at a specific organization), the proposed
protocols offer good protection. 

The knowledge-threshold is an individual user parameter, so users that
consider themselves to be more exposed to risks can choose a higher
knowledge-threshold to increase their protection at the cost of a lower
usability, as a higher threshold makes it harder for other legitimate
users to find them. In that sense, the presented protocols allow users
to individually balance their findability and privacy requirements.



% Communication between people
We will now focus on the communication.
Specifically we will focus on communication between pairs of people, e.g.\ 
Alice talking to Bob.
\citeauthor{otr2004} designed a secure protocol for two-people communication, 
the \ac{OTR} protocol.
They desired an electronic equivalent of face-to-face conversations, i.e.\ that 
they leave no proofs of any kind behind:
if Alice and Bob has had a conversation, Bob cannot go to Eve afterwards and 
prove anything about what Alice has said --- the same as in a face-to-face 
conversation.
This property is not true for email or most centralized communication services.

\subsubsection{Standard Email}

The standard email system does not provide any security.
A suitable analogy would be that each message is a postcard, i.e.\ it has no 
envelope, so the content and address are visible on it.
This means that the postman can read the cards' contents, their recipients' and
senders' addresses.
(Yes, unlike real postcards these also include the sender's address.)
Furthermore, most postmen use transparent sacks to carry the postcards, so 
everyone along the way can also read the sender's and recipient's address and 
the contents.
However, some postmen have started using non-transparent sacks, i.e.\ encrypted 
connections between the servers, so those postcards can only be read by the 
staff in the post-office.
Thus the email system provides no confidentiality: each email server can read 
the messages, each network operator along the transport route can also read 
(and make a copy of) each email.
However, it is actually worse than that, because the email system provides no 
integrity either.
This means that the postman, or anyone along the way, can do arbitrary 
modifications to the messages without anyone noticing the difference.
We can safely say that we cannot rely on the email system for neither security 
nor privacy when planning a protest.

When using a centralized communications service, such as Facebook, the level of
security and privacy we can achieve is that the postman carries non-transparent
sacks.
The business model of most such services is to read peoples postcards to better
profile their interests and thus deliver better suiting advertising.
Here, third parties cannot directly see who is communicating with whom.
They can only see that something goes to and from the service.
However, all information is available internally to the service.
This means that there are ways of learning this, for example through 
PRISM~\cite{Prism} of the \ac{NSA}.

\subsubsection{Secure Email and Text Messaging}

Secure email works by employing cryptography: we encrypt the contents of the 
postcard, providing confidentiality, and then add a digital signature to 
prevent modifications.
Thus the recipient is the only one who can read the message and the recipient 
can also verify that the message has not been modified along the way.
To make key management easy, most schemes use public-key cryptography.
This means that we have two keys, one which is public and another which is kept
private.
For encryption, the public key can transform a message to a ciphertext, i.e.\ 
a random-looking text string.
The private key can be used to transform the ciphertext back to the message.
Given only the public key, it is \enquote{impossible} to find the private key.
For signatures, we can use the private key to compute a signature of a message 
and then send the message and its signature.
The recipient can then use the public key to verify the signature of the 
message.
This signature depends on the entire message, so it is impossible to move 
a signature to another message --- unlike signatures on paper.
And since it is impossible to find the private key given only the public key, 
no one can create fake signatures.

One problem with this approach to secure email is that the sender and recipient
are still in the clear, anyone can read them.
So the content is hidden, but the meta-data is not.

Another problem is that the digital signatures used provides a property called 
non-repudiation.
Say that Alice securely sent an email to Bob, if Eve would compromise Bob's 
private key, as many government agencies can, then she would learn that Alice 
--- and no one else --- has sent that message to Bob.
Bob might even give the message and his key to Eve voluntarily or under threat.
This is exactly the property that \citeauthor{otr2004} wanted to remove with 
\ac{OTR}.
They can do this by leveraging the interactive nature of \ac{IM} and changing 
the digital signatures to shared-key \acp{MAC}.
Shared-key means that Alice and Bob share the same key for generating and 
verifying \iac{MAC}.
This means that Bob can generate valid \acp{MAC} for any message and show to 
Eve, thus he cannot prove to Eve what Alice has said --- since he could have 
created this \enquote{proof} himself.
In addition, Alice and Bob do not use the same \ac{MAC} key throughout their 
conversation, then continuously exchange new keys, one for each message.
However, in this situation, Eve still has only two candidates as the author of 
the message: Alice and Bob, since they both have access to the shared keys.
To remedy this problem Alice and Bob publishes the \ac{MAC} keys after use, 
i.e.\ when they no longer need them.
This gives \enquote{everyone} the possibility of generating messages that 
verifies under Alice and Bob's key, so now Alice and Bob can argue that someone 
(Eve included) could have modified the ciphertext.

The \ac{OTR} protocol became widely spread after the 2013 revelations about the
mass surveillance of the \ac{NSA} and \ac{GCHQ}, many derivatives of the 
protocol emerged in smartphone apps.
Among the most wide-spread derivatives of \ac{OTR} is Signal (formerly 
TextSecure)~\cite{SignalApp}\footnote{%
  TextSecure actually existed before the Snowden revelations, but has seen more
  wide-spread use after.
}.
The Signal protocol has, unlike many other of the derivatives, been formally 
analysed and proven that it indeed provides its claimed security 
properties~\cite{TextSecureAnalysis}.
One improvement over \ac{OTR} is the deniability.
In Signal the authentication is set up in such a way that any person knowing 
the public key of Alice and Bob can generate a fake transcript of 
a conversation.
This results in that Eve has many more candidates for the authors of 
a conversation.

\subsubsection{When the Adversary Controls the Network}

\subsubsection{When the Adversary Controls the Network}
\label{WhenAdversaryControlsNetwork}

\textcite{OTPKX} argue that if the adversary controls the entire network, then 
the approach to deniability taken by \ac{OTR} and Signal does not suffice.
The problem is that Eve can record a transcript of all communications
that have taken place.
We know that the \ac{NSA} did exactly that~\cite{XKeyscore} --- and more 
specifically, saved ciphertexts for later when the decryption key might be 
available.
In this setting it does not matter if anyone can generate a false transcript of 
a conversation between Alice and Bob, because Eve knows exactly what Alice has 
sent, what Bob has received and vice versa.
The argument of this class of protocol is that Alice and Bob have the 
possibility to deny anything about the conversation since it cannot be 
decrypted.
This seems extra problematic when even the free countries in the world suggest 
that there must be ways to break this 
encryption~\cite{BackDoorEncryption}\footnote{%
  We refer the reader to the text by \textcite{KeysUnderDoormats} for further 
  reasons for why this is a bad idea.
}.

There are more than one way to approach this problem.
The first approach would be to use an anonymizing service, such as 
Tor~\cite{Tor}.
This way, Eve would not know that Alice communicates with Bob, only that
Alice communicates with someone.
However, Alice and Bob are located in the same country and Eve controls the 
nationwide network.
For all low-latency anonymizing networks (such as Tor) where the entry point 
and exit are controlled by Eve, Eve can perform a time-correlation 
attack\footnote{%
  This means that Eve records the time of when each message enters the network 
  (entry distribution) and the time when each message exits the network (exit 
  distribution).
  Due to the low-latency property, these distributions will be related and Eve 
  can infer to whom Alice sent her message.
} and essentially render the anonymization service 
useless~\cite{SystemsForAnonymousCommunication}.
To make this attack more difficult for Eve, the system must introduce random 
delays in our communication\footnote{%
  The delays must transform the exit distribution to a distribution more 
  similar to the uniform distribution, then Eve's statistical analysis will 
  become more difficult.
}.
(We will return to this topic in \cref{MessageDistribution}.)
But despite all this, Eve can still ask Alice to decrypt the conversations, 
either she complies or claims that she does not know the key.

The second approach would be to ensure deniability even against this strong 
adversary.
This would not hide who communicates with whom, as in our first approach, but 
it provides deniability for the conversations.
The scheme suggested by \textcite{OTPKX} makes use of one practical instance of 
deniable encryption~\cite{DeniableEncryption}.
They construct a scheme where Alice and Bob can create \enquote{false proofs} 
for their conversation.
In essence, Eve records all traffic.
When she approaches Alice and asks her to provide a key to decrypt the recorded 
traffic, Alice can create a decryption key such that when Eve decrypts the 
recorded traffic will receive a plaintext of Alice's choice.
This way Alice can plausibly deny any allegations.
However, the question whether Eve would actually accept such a \enquote{proof}, 
knowing it might equally well be false, remains open.



% Holding discussions
So far we have treated only one-to-one conversations, i.e.\ Alice and Bob 
talking to each other.
However, there are usually more than two people organizing a protest, and so we 
need to hold discussions with more than only two people at a time.
In this situation there are two approaches to solving the communication:
simultaneous pair-wise communication between all participants or true group 
communication.
Furthermore, how the messages are distributed is also important, because Eve 
can learn who the participants are.

\subsubsection{Group Communication Properties}
\label{GroupProperties}

When a group uses pair-wise communication, every member of the group will set 
up a pair-wise channel to each other member of the group.
Each pair-wise channel is as described above, in \cref{Communicating}.
Then for every message Alice wants to send to the group she has to send it to 
every participant.
This would allow Alice to cheat, e.g.\ she can send \enquote{Who wants to 
  overthrow the regime?} to everyone except to Bob, to whom she instead sends 
\enquote{Who wants to order pizza?}.
This opens up for the Byzantine Generals' problem~\cite{ByzantineGenerals}, 
where malicious actors can lie to honest actors to disrupt operation.
\citet{ByzantineGenerals} in fact proved that it is impossible for the honest 
parties to recover and identify the malicious parties if the malicious parties 
exceed a third of the participants.
  
Although Alice's ability to say different things to different participants is 
in itself a desirable property from Alice's perspective --- she would like to 
lie to suspected regime agents --- this property can at the same time be 
undesirable due to the Byzantine Generals' problem.
For this reason group communication must provide better properties, namely that 
everyone hears who said what and when, thus forcing Alice to say the same thing 
to all participants.
In such a scheme, when Bob replies \enquote{I do, shall we say tonight?} the 
others will see that Bob is replying to something they did not see and not to 
the question \enquote{Who wants to overthrow the regime?}.

\citet{multiotr2009} tried to extend the \ac{OTR} protocol to a multi-party 
setting.
This did not result in a concrete protocol implementation, and the resulting 
protocol they suggested was also very complex.
It also had some undesirable limitations, for instance, the scenario that Bob 
receives a question which is different from everyone else's is only detected at 
the very end of the conversation.
As is pointed out by \citet{TSgroups}, asynchronous communication today has no 
real end, which makes the approach of \citet{multiotr2009} even less appealing.
Due to this, \citet{SignalApp} implements group chats as simple pair-wise 
conversations.
With additional meta-data they can ensure consistent history, however, this is 
not yet implemented~\cite{TSgroups}.
A technique that could be used for this is to include a message digest of the 
entire conversation history with each message.
A message digest is computed using a one-way function, i.e.\ its output is 
unpredictable and its input is impossible to compute given only the
output. %TODO: explain more clearly
This means that the message digest included in Bob's reply and the one computed 
by the other participants above would differ, thus everyone learns that the 
conversation history is inconsistent and should no longer be trusted.
Due to the unpredictable property of the one-way function, Alice cannot phrase 
the two different messages in such a way that they yield the same message 
digest in the history either.
But despite this, the other participants cannot determine if it is Alice or Bob
who is lying about the message history --- Alice could send the same message to
everyone and still Bob could try to frame her.

\subsubsection{Message Distribution}
\label{MessageDistribution}

\textcite{PPACinPubFS} analysed two dichotomous models of communication: the 
pull model and the push model.
Alice wants to send a message to Bob and Carol.
In the pull model, Alice would publish her message in a place known to Bob and 
Carol.
Bob and Carol visits this place periodically to check if Alice has published 
any new messages.
(If there is a new message, they make a copy for themselves.)
In the push model, Alice drops her message in Bob's and Carol's letter boxes.
(One copy for each.)
Email and text-messaging are best modelled using the push model (see 
\cref{GroupCommunication}).
For both models, Eve can analyse the communication patterns to infer (a part 
of) the social graph unless Alice, Bob and Carol use some countermeasures.

%\citeauthor{PPACinPubFS} found that achieving privacy in the pull model is 
%technically easier than in the push model.
%In fact, achieving strong privacy in the push model is very
%difficult. %TODO: explain why (move to end)

\paragraph{The Pull Model}

We can start by looking at the pull model for communication.
Alice wants to distribute a message to Bob and Carol, the participants in 
a discussion.
In the pull model they actively ask Alice (or an intermediary) for new messages 
at regular intervals.
To form a protocol around this model, Alice, Bob and Carol can agree on 
a location where Alice puts her messages\footnote{%
  Technically, this can be implemented in a similar fashion as the \ac{DHT}, as 
  mentioned in \cref{UserSearch}.
  This would make it more difficult for Eve to censor it compared to the 
  centralized systems.
}.
When Alice wants to send a new message, she writes it to this particular 
location.
When Bob and Carol want to, they can read from the location to see if there are 
any new messages.

Suppose that Eve controls the network that Alice uses\footnote{%
  This is reasonable considering that we saw earlier that the \ac{NSA} and 
  \ac{GCHQ} achieve similar results in free countries.
}.
Since we have a decentralized system in mind, we can also assume that anyone 
(especially Eve) can read anything from the storage.
%This is why Alice encrypts all her messages for the recipients' keys.
%Also, Alice does not want to be associated with the message, not authorship nor 
%posting it.

The first thing we can say about this situation is that Alice would like to 
have confidentiality for the messages' contents, so that Eve cannot read her 
messages.
Alice would also like to have integrity for her messages, so that Bob and Carol 
can be sure that they are from Alice and that Eve has not modified them.
Many systems provide these two properties, e.g.\ \ac{PGP} does this for email 
(and could be applied here as well).
However, Alice also wants to hide the sender and recipients, which many systems 
(including \ac{PGP}) do not provide.
There is a class of encryption schemes called \ac{ANOBE} schemes.
This type of scheme provides confidentiality while hiding the sender and the 
intended recipients.
If Alice can write the message anonymously to the storage and the message is 
encrypted using \iac{ANOBE} scheme, then it will be difficult to determine the 
sender.
Furthermore, if the recipients fetch the messages anonymously too, then the 
recipients are also hidden.
The idea is as follows: if Eve cannot distinguish between Bob and Carol 
fetching a message, then it might just as well only be Bob who fetches messages 
from this location --- Eve cannot tell the difference.
(We will return to this problem later.)

The problem of integrity remains.
There are two approaches: digital signatures and \acp{MAC}.
If Alice, Bob and Carol agree on a commonly shared \ac{MAC} key, then they can 
use \acp{MAC} to ensure integrity.
One advantage of \acp{MAC} is that anyone who can verify the authenticity of 
\iac{MAC} can also create one (as was pointed out above).
With digital signatures on the other hand, if Alice signs a message it is clear 
that Alice is the only one who could have signed it.
(It is important that only Bob and Carol know that Alice owns the private key, 
and that it remains anonymous to Eve.)
But this provides Eve with something to track messages by, all messages signed 
by the same key are related.
With \acp{MAC}, Bob and Carol could also have authored the message and Eve 
cannot determine which messages are related either.
This means that for a discussion, any of the participants would be equally 
likely to be the author of a given message.
However, this relies on the anonymity of the actors.

\paragraph{The Push Model}

In the push model, Bob and Carol have one location each where Alice will drop 
her messages.
(She can achieve confidentiality and integrity similarly as in the pull model.)
One thing we can observe is that the recipients are not as hidden as in the 
pull model, even if we assume anonymity.
Eve can observe that someone (Alice, but Eve does not know that) puts two 
messages at the same time.
Eve can then observe these locations to see when someone (Bob or Carol, but Eve
does not know that either) reads messages from those locations.
The main problem with the push model is that it reveals more meta-information 
than the pull model does.
With the push model Eve can build herself a map of the social graph\footnote{%
  Since Eve works with probability distributions, this would be an 
  approximation of the social graph.
  But her approximation can come very close to the real one.
}.
Then she only needs to map the real identities of Alice, Bob and Carol to these 
anonymous identities.

\paragraph{Privacy}

%Say that Alice, Bob and Carol have one inbox each, similarly as in the email 
%system or Signal.
%Eve monitors the network on a national level.
%Now Eve can see one message originating from Alice, going to a server beyond 
%Eve's reach, and soon two equally-sized messages return from the server 
%near-simultaneously to Bob and Carol\footnote{%
%  Or equivalently, Eve observes where these messages end up in the storage and 
%  later observes Bob and Carol fetching these messages.
%}.
%(As we pointed out above, this is what is called a time-correlation attack.)
%This is what happens when Alice, Bob and Carol do not have perfect anonymity, 
%e.g.\ by using Tor or simply using Signal without any anonymizing service.
%Eve can relate Alice, Bob and Carol to each other.
%Now let us try to make this more difficult for Eve.
%
Say that Alice, Bob and Carol can access the storage system for messages 
anonymously, i.e.\ Eve can only observe when Alice, Bob and Carol does 
something --- but not what they do --- and when something happens in the 
storage system --- but not who does it.
Despite this anonymity, Eve can still do a correlation attack.
For example, Eve can temporarily detain Bob (or turn off his network 
connection) and observe the change in the distribution of reads from the 
storage.
In the push model, she will observe that someone stopped reading from one of 
the locations, i.e.\ Bob's location.
The same argument can be applied to the pull model case: when Eve detains Bob, 
she can observe a change in the probability distribution of reads from where 
Alice puts her messages.
In fact, even if several people share locations, this simply slow Eve down.

As was pointed out above (\cref{WhenAdversaryControlsNetwork}), the solution to 
this type of attack is to add noise to make these changes in distribution 
indistinguishable.
\citeauthor{PPACinPubFS} suggested that differential privacy\footnote{%
  Differential privacy guarantees that if we remove any \emph{one} data item, 
  the distribution will change below a given threshold.
} will probably be the best trade-off between privacy and efficiency.
In fact, parallel to the work of \citeauthor{PPACinPubFS}, 
\textcite{Vuvuzela,Alpenhorn} designed a protocol for one-to-one communication 
based on differential privacy.
There are still some limitations, e.g.\ everyone must be online and 
participating in the protocol all the time --- 24 hours every day.

%If one of the participants makes a mistake, then the regime's agents will have 
%a starting point to target.
%For example if a participant uses the same inbox for communication with all his 
%friends, and not only the participants in the plot against the regime, then one 
%of his other contacts might not be as concerned with staying anonymous.
%The consequence is that the regime can see the identity of someone sending 
%messages to an inbox of their interest.
%Then they can target this person and learn which friend owns the inbox of 
%interest.
%Then they can proceed to targeting one of the protest organizers.
%This type of attack will not work when the communication is according to the 
%pull model, since there the agency must attack each anonymous connection.
%%TODO: reviewer: explain what you mean by attack here
%




% Inviting participants
There are some tasks that the organizers must accomplish prior to the protest 
itself.
For example, they must decide who are the most suitable candidates to attend 
the event, how to let them know about the protest and what preliminary 
information they should get.
They must also decide whether invitees should learn about the attendance of 
other invitees or not.
Preferably, all this should be possible in a privacy-preserving
fashion. %TODO: reviewer: unclear

Realizing this standard feature of \acp{OSN} in a decentralized manner is not 
trivial, because there is no trusted third party which both organizers and 
invitees can rely on. %TODO: reviewer comment: give more concrete
                      %examples, like FB, analyze why FB has flaws
Since they %TODO: who?
 all depend on only themselves, an implementation of this feature 
must provide security properties that guarantee fairness to all parties 
involved, e.g.\  a protester can verify that the invitation she received was 
actually sent by the organizers.
Moreover, the implementation should also provide privacy settings to protect 
personal information such as the identities of the participants, e.g.\ the 
organizers can restrict to only the invited participants to learn how many 
others have been invited, and only after a protester has agreed and committed 
formally to attend the event, she can learn the identities of other invited 
protesters.

The challenge of implementing this feature without a trusted third party 
becomes greater when the organizers want to allow different types of 
information about the event to be shared with different groups of protesters in 
a secure way because any participant should be able to verify the results to 
detect any possible cheating.
For example, a neutral trusted broker, such as the organizers, could keep 
certain information secret, such as the identities of the invited protesters, 
and only disclose it to those ones who commit to attend the protest. 

In the scheme by \citet{EventsInvitations}, they describe and formalize the 
security and privacy properties outlined above.
More specifically, that the organizer is able to configure who can learn the 
identities of the invited or attending participants, or a more restrictive 
version where only the number of invitees or attendees are revealed.
There is also an attendee-only property that guarantees exclusive access to 
some data, e.g.\ the location of the protest.
These properties are accomplished using several simple primitives:
storage location indirection, controlled ciphertext inference and 
a commit-disclose protocol.
Storage location indirection allows the organizer to control not only who can 
read some the contents of some encrypted data, but also who can access the 
ciphertext.
% XXX Why is this interesting?
Controlled ciphertext inference can be used by the organizer to allow for 
controlled information leaks.
This is needed to achieve properties such as revealing the number of invitees 
but keeping their identities secret.
Finally, the commit-disclose protocol can make some secret available for only 
those participants who committed to attend the protest while, at the same time, 
detect any misbehaving party.


% Having a Decentralized Account
%\subsection{Having a Decentralized Account}

In general, there are some more properties that are relevant for these systems.
Some secure authentication schemes allows for \ac{DoS} attacks against the 
proper account holder.
Although it prevents the attacker from gaining access, it also prevents the 
authentic user:
e.g.\ what should happen if multiple users try to access an account at the same 
time?
If the intelligence services can lock the activists out of their accounts and 
thus forcing them to resolve to less secure means of communication, then the 
intelligence services have won.
\citet{P2PPasswords} developed mechanisms towards solving this problem.


