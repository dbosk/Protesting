\section{Before a Protest}
\label{BeforeProtest}

There are several issues related to protesting in the stage before.
First comes the initial discussion between potential organizers, second comes 
the problem of scheduling this with participants.
For the initial discussion, the potential organizers don't want the regime's 
intelligence services to identify them as such.
One of the most popular secure-messaging protocols is \ac{OTR}, lately 
popularized in smartphones through Signal (formerly TextSecure).
\citet{OTPKX} proposed a scheme with the same properties as \ac{OTR} but with 
more deniability.
Another problem in this stage is to find other's user profiles.
\citet{ThresholdUserSearch} designed a scheme for targeted user search by means 
of user defined knowledge threshold.
Protesters must be able to find each other in the networks, but we don't want 
the oppressive regime to do the same.

For the scheduling of a protest, there are in turn several problems that must 
be addressed.
From the organizer Alice's perspective, she wants to protect herself from being 
arrested for organizing a protest.
So Alice needs to protect herself from the possible participants, as one of 
them can be agent Eve of the intelligence services of the regime.
From the participant Bob's perspective, he wants to protect himself from being 
arrested for committing to participate in a protest.
So Bob needs to protect himself from the organizer and the other participants, 
as any of them can be Eve.

When organizing a protest, what Alice and Bob want to agree on is a time, 
a place and to ensure enough people will show up at that time and place.
Alice and Bob also wants Eve to learn as little as possible of the plans so 
that she cannot curtail the protest.

\citet{EventsInvitations} presented a distributed protocol without the need of 
\iac{TTP}.
This protocols allows for different privacy settings:
\begin{itemize}
\item Alice discloses nothing to Bob, except the time and the place;
\item Alice discloses everything --- who the invitees are, who of those have 
  already committed etc.;
\item and every combination of settings in between.
\end{itemize}
Further, if Alice doesn't keep her promises, Bob has a proof which he can 
publish to everyone to show that Alice cheated.
Likewise, if Bob commits to attending, Alice has proof that Bob has done so and 
can show to everyone that Bob isn't present although he said he would.

One topic that must be explored is to adapt this protocol to introduce 
deniability.
Another interesting feature to include would be the choice of location.
With this, all participants can jointly agree on not only a time, but also 
a location.
This would help against the problem of announcing the location in advance.

