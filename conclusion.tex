\section{Conclusions}
\label{Conclusions}
%TODO: reviewer: the last part, after protest" is not really about
%enhancing privacy or about surveillance
In this paper we presented some privacy-enhancing technologies that can be 
relevant for political activism, with a focus on supporting various kinds of 
protests from online organization to evaluation after the fact. While we 
discussed a range of technologies, this selection presents only a small part of 
what has been conceived of in the greater research community over the last
decades. In addition to privacy-enhancing technologies, there has also
been research on transparancy-enhancing technologies (TETs), that
focus on revealing how data is collected and treated - an important
perspective on privacy that we deemed out of scope for this chapter.

There 
still is a large gap to bridge between privacy researchers and political 
activists. As in the general population, not many of the existing technologies 
for privacy are actually used, and those that are used (Tor being such 
a notable exception) are not often used widely. There are several reasons for 
this, in our opinion. The main reason is perhaps a lack of communication. It is 
difficult to communicate across different academic disciplines and between 
academia and the outside world. People may not have heard of relevant 
technologies, may be reluctant to use them, or may not see their use case 
reflected. The so-called engineer's disease of offering technical solutions to 
social problems is reflected in the quite common lack of thorough analysis of 
what actual users need; we need a dialog to channel technical
approaches to appropriate uses. Usability is another issue that will need to be taken 
into account more, requiring interdisciplinary work, communication, and user 
participation. Many of the existing technologies, while conceptually excellent, 
only exist as prototypes, or as isolated primitives that are not integrated 
into a usable and comprehensive service. 

There is much work left to connect the 
existing technologies with actual people that can benefit from them, and vice 
versa, researchers lack input on what is actually needed in the field and often 
scalable ways of turning academic prototypes into complete and secure tools to 
protect the privacy of the people.
