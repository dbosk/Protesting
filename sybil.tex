The lack of a trusted third-party when taking away the central provider in a decentralized 
network is susceptible to \emph{Sybil} attacks. 
In such situation, a rather small number of people in the network control a big 
set of the identities compromising a large portion of the network in order to gain 
a disproportionate amount of influence~\cite{SybilAttack}.

The problem of \emph{Sybil} attacks can not be trivially solved by having the people 
who are already in the network to vouch for other identities, in a network of trust 
manner. The attack itself aims at compromising a considerable portion of the identities 
in the space reducing the vouching chain.

Although the \emph{Sybil} attack problem remains unsolved to date without a central trusted 
broker that acts as an identification authority establishing a one-to-one correspondence 
between an identity and a user, there are methods and techniques in the literature 
to mitigate its effect. 

% TODO Reprhase and complete the following
%In a large-scale scenario, like a decentralized network, is nearly impossible to 
%establish distinct identities without a central authority vouching for these identities.
%Entities can validate in two ways:
%Direct identity validation
%Indirect identity validation

\dots
