The Sybil attack is somewhat related to the double agent problem, but is only 
a problem in electronic systems.
The problem occurs when there is nothing that limits the creation of new 
identities, thus the adversary can create multiple unlinkable identities.
Usually the attack is aimed at reputation systems, where the adversary can use 
its many identities to vouch for each other to falsely gain in reputation.
In general, the problem can be summarized as that a rather small number of 
people in the network control a large part of the identities in the network in 
order to gain a disproportionate amount of influence~\cite{SybilAttack}.

\Citet{SybilAttack} proved that this problem cannot be solved without 
a \emph{logically} central control of the the creation of identities.
This means that we can not handle identity creation by letting the people who 
are already in the network vouch for other identities, i.e.\ to build a network 
of trust.
The Sybil attack itself aims at compromising a considerable portion of the 
identities, and thus, the more identities the attacker gains, the more new 
identities it can create and vouch for.
To prevent this kind of behaviour we rather need something like the national 
identification systems present in most countries, where the state has ensured 
a one-to-one correspondence between identities and physical persons.
Fortunately, there are techniques that can mitigate the effects of the Sybil 
attack without forcing us to use such a centralized identity system.
We will return to these where relevant in the text.
%TODO make sure we do

%This is, however, less desirable in the scenario we consider here.
%Fortunately, if we use techniques like \ac{MPC}, then we can overcome the 
%limitation of having a central authority.
%With \ac{MPC} we can construct a logical trusted third-party by having the 
%participants run a certain protocol, i.e.\ all participants constitute the.

%The Sybil attack problem remains unsolved without a central trusted broker that 
%acts as an identification authority, but there are methods and techniques in 
%the literature to mitigate its effect.
%We will return to these where it is relevant in the rest of the text.

