\section{After a Protest}
\label{AfterProtest}

There are at least two things that are interesting after a protest:
\begin{inparaenum}[(a)]
\item that a third-party can verify the authenticity of the participation of 
  a protest;
\item for the organizers to use the participation as feedback into a reputation 
  system.
\end{inparaenum}
We want to verify the participation of a protest, but not identifying 
individuals who participated.
E.g.\ we want to compute the number of participants to verify the scale of the 
protest.
This can be useful for e.g.\ the UN to verify protests happening in a country; 
the country cannot deny it and the UN can apply pressure if needed.

Purely online protests are essentially petitions.
This can be compared to electronic voting --- to vote for or against something.
For this purpose we should be able to adapt existing electronic voting 
protocols.
In this case all participants and third-parties can verify the authenticity of 
the result, but cannot verify the vote of an individual.

For real-world protests we need to bind participants to the same physical 
location at a reasonably similar time (within the duration of the protest).
\citet{PROPS} developed a decentralized location-proof system which provides 
a participant with a proof of being at the location.
However, we would like that any participant can prove the demonstration's 
authenticity to a third-party, the scenario for this would be that only one 
demonstrator survives and manages to flee the country.

This location-proof system should be possible to combine with the scheduling 
system in \cref{BeforeProtest} for feedback into a reputation system.
One problem in these distributed systems is the possibility for the 
intelligence services to create multiple identities and thus perform a Sybil 
attack~\cite{SybilAttack} on the activists.

\citet{PPACforPubFS} presents work done in the area of privacy-preserving 
access control in distributed storage systems.
This is important since it outlines some possibilities and limits for such 
systems.
The problem in this scenario is that we don't want to be identifiable as 
a demonstrator, as this might yield repercussions.
Thus we can to share the data anonymously, no one can tell with whom we've 
shared what or if they've read it.
However, we still want to verify authenticity as it might otherwise be the 
regime spreading disinformation.

In this case it is also not straight forward to just apply the techniques in 
\cite{OTPKX} to achieve deniability.

