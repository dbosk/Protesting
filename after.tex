\section{After a Protest}
\label{AfterProtest}

As we pointed out earlier, the main goal of the stage after a protest is to 
provide verifiable data.
For instance, how can we ensure that photos from a demonstration are authentic?
We can probably recognize the place the photo is portraying, however, the time 
data of the file can be manipulated.
Similarly, how can we determine the number of participants of a demonstration?
We might be able to estimate the number given photos of the demonstration, but 
this might not give an accurate number, especially since the authenticity of 
the photos can be questioned.
We will discuss two problems in this section:
\begin{enumerate}
  \item ensuring data authenticity related to a demonstration, e.g.\ so that 
    the data can be correctly tied to the demonstration;
  %\item the possibilities for the organizers to verify the participation and 
  %use it as feedback into a reputation system;
  \item third-party verification of the authenticity of the participation of 
    a protest, e.g.\ compute the number of participants.
\end{enumerate}

The first problem will be discussed in \cref{DataAuthenticicty} and the second 
in \cref{ParticipationAuthenticity}.

\subsection{Ensuring Data Authenticity}
\label{DataAuthenticicty}

The problem of authentically associating data with a physical event is 
difficult.
We can essentially divide it into the following requirements:
\begin{requirements}
  \item\label{CreatedBeforeEnd} Prove that the data was created before the end 
    of the event.
  \item\label{CreatedAfterStart} Prove that the data was created after the 
    start of the event.
  \item\label{SpatiallyRelated} Prove that the data is spatially related to 
    the physical location of the event.
\end{requirements}
\Cref{CreatedBeforeEnd,CreatedAfterStart} together bind the data to the time of 
the event whereas \cref{SpatiallyRelated} binds the data spatially to the 
event.

Now, consider the scenario of Alice taking a photo during the demonstration and 
posting it online.
What can we say about this photo?
First, we can say that it was created before we viewed it, so 
\cref{CreatedBeforeEnd} is fulfilled if we view it in relation to the event.
If it was submitted to a service that we trust, then we can also trust the 
time-stamp of the service.
Furthermore, we can also consider it spatially related to the physical location
(\cref{SpatiallyRelated}) if we can convince ourselves that the photo is 
depicting the physical location and not any kind of \enquote{reconstruction}, 
e.g.\ it is computer generated or a photo of a similarly looking location.

\Cref{CreatedAfterStart} is more difficult to achieve.
In the above scenario, there is nothing preventing the photo from being much 
older than the event yet spatially related to the physical location.
In the security field \cref{CreatedAfterStart} is captured by a property called 
\emph{freshness}.
The freshness property is usually achieved by requiring that the data depend on
an unpredictable value.
The unpredictable value is commonly a value chosen randomly by the verifier, 
but the main basic requirement is that it is not under the prover's control, 
i.e.\ Alice in this scenario.
For example, we might require Alice to include the front-page of a particular 
newspaper in the photo, since the exact front-page is difficult for Alice to 
predict in advance.
There is a subfield of digital forensics that work with image manipulation 
detection, so there are methods that would prevent at least Alice's easiest 
manipulation attempts.

We can see that the techniques to achieve 
\cref{CreatedBeforeEnd,CreatedAfterStart,SpatiallyRelated} depends on the type 
of data.
We used as an example above a photo, next we will look at another type of data.


\subsection{Third-Party Verification of Protest Participation}
\label{ParticipationAuthenticity}

We want to verify the participation of a protest, but not identifying 
individuals who participated.
E.g.\ we want to compute the number of participants to verify the scale of the 
protest.
This can be useful for e.g.\ the UN to verify protests happening in a country; 
the country cannot deny it and the UN can apply pressure if needed.

Purely online protests are essentially petitions.
This can be compared to electronic voting --- to vote for or against something.
For this purpose we should be able to adapt existing electronic voting 
protocols.
In this case all participants and third-parties can verify the authenticity of 
the result, but cannot verify the vote of an individual.

For real-world protests we need to bind participants to the same physical 
location at a reasonably similar time (within the duration of the protest).
\citet{PROPS} developed a decentralized location-proof system which provides 
a participant with a proof of being at the location.
However, we would like that any participant can prove the demonstration's 
authenticity to a third-party, the scenario for this would be that only one 
demonstrator survives and manages to flee the country.

This location-proof system should be possible to combine with the scheduling 
system in \cref{BeforeProtest} for feedback into a reputation system.
One problem in these distributed systems is the possibility for the 
intelligence services to create multiple identities and thus perform a Sybil 
attack~\cite{SybilAttack} on the activists.


