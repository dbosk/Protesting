\section{After a Protest}
\label{AfterProtest}

As we pointed out earlier, the main goal of the stage after a protest is to 
provide verifiable data.
For instance, how can we ensure that photos from a demonstration are authentic?
We can probably recognize the place the photo is portraying, however, the time 
data of the file can be manipulated.
Similarly, how can we determine the number of participants of a demonstration?
We might be able to estimate the number given photos of the demonstration, but 
this might not give an accurate number, especially since the authenticity of 
the photos can be questioned.
We will discuss two problems in this section:
\begin{enumerate}
  \item ensuring data authenticity related to a demonstration, e.g.\ so that 
    the data can be correctly tied to the demonstration;
  %\item the possibilities for the organizers to verify the participation and 
  %use it as feedback into a reputation system;
  \item third-party verification of the authenticity of the participation of 
    a protest, e.g.\ compute the number of participants.
\end{enumerate}

These problems will be discussed in \cref{DataAuthenticicty,ProtestVerif}, 
respectively.

\subsection{Ensuring Data Authenticity}
\label{DataAuthenticicty}

The problem of authentically associating data with a physical event is 
difficult.
We can essentially divide it into the following requirements:
\begin{requirements}
  \item\label{CreatedBeforeEnd} Prove that the data was created before the end 
    of the event.
  \item\label{CreatedAfterStart} Prove that the data was created after the 
    start of the event.
  \item\label{SpatiallyRelated} Prove that the data is spatially related to 
    the physical location of the event.
\end{requirements}
\Cref{CreatedBeforeEnd,CreatedAfterStart} together bind the data to the time of 
the event whereas \cref{SpatiallyRelated} binds the data spatially to the 
event.

Now, consider the scenario of Alice taking a photo during the demonstration and 
posting it online.
What can we say about this photo?
First, we can say that it was created before we viewed it, so 
\cref{CreatedBeforeEnd} is fulfilled if we view it in relation to the event.
If it was submitted to a service that we trust, then we can also trust the 
time-stamp of the service.
Furthermore, we can also consider it spatially related to the physical location
(\cref{SpatiallyRelated}) if we can convince ourselves that the photo is 
depicting the physical location and not any kind of \enquote{reconstruction}, 
e.g.\ it is computer generated or a photo of a similarly looking location.

\Cref{CreatedAfterStart} is more difficult to achieve.
In the above scenario, there is nothing preventing the photo from being much 
older than the event yet spatially related to the physical location.
In the security field \cref{CreatedAfterStart} is captured by a property called 
\emph{freshness}.
The freshness property is usually achieved by requiring that the data depend on
an unpredictable value.
The unpredictable value is commonly a value chosen randomly by the verifier, 
but the main basic requirement is that it is not under the prover's control, 
i.e.\ Alice in this scenario.
For example, we might require Alice to include the front-page of a particular 
newspaper in the photo, since the exact front-page is difficult for Alice to 
predict in advance.
There is a subfield of digital forensics that work with image manipulation 
detection, so there are methods that would prevent at least Alice's easiest 
manipulation attempts.

We can see that the techniques to achieve 
\cref{CreatedBeforeEnd,CreatedAfterStart,SpatiallyRelated} depends on the type 
of data.
We used as an example above a photo, next we will look at another type of data.


\subsection{Verification of Protest Participation}
\label{ProtestVerif}

One problem in real-world demonstrations that has not yet been entirely solved 
is the crowd counting problem, i.e.\ the verification of the participation in 
a protest.
After many protests the demonstrator count by police and that by organizers 
differ, in some instances the difference can be hundreds of thousands.
There are numerous examples, e.g.\ the demonstrations in South 
Korea~\cite{2016DemonstrationsInSeoul} or the women's maches in the 
US~\cite{2017WomensMarchesInUS}, where there is difficulty in establishing the 
actual number of participants.
The methods for counting the crowds vary.
There are some techniques using computer vision found in the research 
literature, e.g.\ by~\textcite{CVCrowdCounting}.
These require images from the demonstration and provide no way to verify the 
authenticity of the count (except re-running the algorithm on one's own input 
data).
However, as is illustrated by 
\textcite{2016DemonstrationsInSeoul,2017WomensMarchesInUS}, the methods applied
are manual and prone to errors.

In general, a demonstration is very similar to voting: both are many 
individuals expressing their opinion.
Hence it is desirable to have similar properties for verifying the 
participation in a protest, where this verification step is at the core.
In the context of (electronic) voting protocols, there are three desirable 
properties for verification:
\begin{requirements}[V]
\item\label{EligibilityVerif} Eligibility: anyone can verify that each vote 
  cast is legitimate.
\item\label{UniversalVerif} Universal verifiability: anyone can verify that the 
  result is according to the cast votes.
\item\label{IndividualVerif} Individual verifiability: every voter can verify 
  that its vote is included in the result.
\end{requirements}
We can translate these properties to requirements for the case of protest 
participation, then each vote would be replaced by a proof of participation (in 
either the protest or any counter-protest).
\Cref{EligibilityVerif} would in this case mean that anyone can verify that 
each participation proof belongs to a unique individual, i.e.\ to prevent any 
Sybil attack (as discussed in \cref{SybilAttacks}).

The three verifiability properties above are indeed desirable, e.g.\ then the 
\ac{UN} can verify protests happening in a country and the country cannot deny 
it, thus the \ac{UN} can apply pressure if needed.
However, the properties are difficult to accomplish with computer vision 
methods, especially if nobody from the \ac{UN} has been on location to collect 
the image material for the analysis --- otherwise the authenticity can be 
questioned (as discussed in \cref{DataAuthenticity}).

We also need privacy in addition to the verification requirements.
In voting, we have the properties:
\begin{requirements}[P]
\item\label{VotePrivacy} Vote privacy: the vote does not reveal any individual 
  vote.
\item\label{ReceiptFreeness} Receipt freeness: the voting system does not 
  provide any data that can be used as a proof of how the voter voted.
\item\label{CoercionResistance} Coercion resistance: a voter cannot cooperate 
  with a coercer to prove the vote was cast in any particular way.
\end{requirements}
\Textcite{VerifyingPrivacyPropertiesOfVotingProtocols} showed that 
\cref{CoercionResistance} implies \cref{ReceiptFreeness}, which in turn implies
\cref{VotePrivacy}.
We can see that \cref{VotePrivacy,ReceiptFreeness} would be interesting in the 
context of a demonstration: there should not be any proof which binds Alice to 
the participation of the demonstration (\cref{ReceiptFreeness}), because she 
does not want to explicitly reveal her opinions to the government 
(\cref{VotePrivacy}) due to the risk of reprimands.

Additionally, in voting, the cast votes are not linkable to the identity of the 
casters.
Thus it is not a problem to reveal the identites of those who participated in 
the vote, since they could have voted for any alternative.
We would also like to have the corresponding property when verifying the 
protest.
The very nature of a protest is different though, we do not have any choice: if 
we participate we support the cause of the protest.
%TODO: what about counterprotests in the same location
Consequently we want to verify the participation of a protest without 
identifying individuals who participated.
Otherwise the regime's agents can identity all the participants and simply 
\enquote{make them disappear}. %TODO: reviewer: unclear

Finally, for real-world protests we need to bind participants to the same 
physical location at a reasonably similar time, i.e.\ within the area and 
duration of the protest.
This means that for a system like this to work, we also need the requirements 
from \cref{DataAuthenticity}: proof that the data was created after the start 
of the protest, proof that it was created before the end, and proof of spatial 
relation to the location.
For the last property, \citet{PROPS} developed a decentralized \ac{LPS} which 
provides a participant with a verifiable proof of having been at a location at 
a certain time, something we call \iac{LP}.
It is decentralized because there is no central authority that vouches for the 
location, instead peers act as witnesses.
Then a third-party can verify the authenticity of the \ac{LP}, by verifying the 
witnesses signatures, and can thus be sure that the person has indeed been in 
the location.
Bosk, Gambs and Buchegger are currently exploring (work in progress) the 
possibility of combining such \iac{LPS} with the verifiability properties for 
voting protocols into a system for verifying the participation in protests.
The overall idea is that each participant generates \iac{LP} during the 
protest, where (some of) the other protesters act as witnesses, then the 
\acp{LP} can be used to compute the participation count with all the 
verifiability properties above.
%TODO: reviewer: what about cellphones to count attendance

