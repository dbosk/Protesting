\section{Technical limitations}
\label{TechnicalLimitations}

We can only do so much with technology, and, of course, there are some 
limitations to what technical solutions can achieve.
In particular, we face two problems.
The first one is the double agent problem.
It is caused by humans ability to deceive each other, and consequently cannot 
easily be solved by technology.
\label{DoubleAgentProblem}
In out context it is the problem of one of the regime's agents infiltrating the 
opposition by acting as if part of the opposition.
We cannot solve this problem, however, we might be able to reduce the damage.
One design principle for privacy is data minimization, this strategy will help 
Alice reduce the information that the double agent, and anyone else, can learn.

The second problem is the Sybil attack.
This problem is a consequence of the unlimited ability of creating arbitrary 
identities in online systems such as social networks.
\label{SybilAttacks}
The Sybil attack is somewhat related to the double agent problem, but is only 
a problem in electronic systems.
The problem occurs when there is nothing that limits the creation of new 
identities, thus the adversary can create multiple unlinkable identities.
Usually the attack is aimed at reputation systems, where the adversary can use 
its many identities to vouch for each other to falsely gain in reputation.
In general, the problem can be summarized as that a rather small number of 
people in the network control a large part of the identities in the network in 
order to gain a disproportionate amount of influence~\cite{SybilAttack}.
A concrete example is Trump's use of Twitter bots in the US 
election~\cite{BotsAndAutomationDuringUSElection}.

\Citet{SybilAttack} proved that this problem cannot be solved without 
a \emph{logically} central control of the creation of identities.
This means that we can not handle identity creation by letting the people who 
are already in the network vouch for other identities, i.e.\ to build a network 
of trust.
The Sybil attack itself aims at compromising a considerable portion of the 
identities, and thus, the more identities the attacker gains, the more new 
identities it can create and vouch for.
To prevent this kind of behaviour we rather need something like the national 
identification systems present in most countries, where the state has ensured 
a one-to-one correspondence between identities and physical persons.
Fortunately, there are techniques that can mitigate the effects of the Sybil 
attack without forcing us to use such a centralized identity system.
We will return to these where relevant in the text.
