We will now focus on the communication.
Specifically we will focus on communication between pairs of people, e.g.\ 
Alice talking to Bob.
\citeauthor{otr2004} designed a secure protocol for two people communication, 
the \ac{OTR} protocol.
They desired an electronic equivalent of face-to-face conversations, i.e.\ that 
they leave no proofs of any kind behind:
if Alice and Bob has had a conversation, Bob cannot go to Eve afterwards and 
prove anything about what Alice has said --- the same as in a face-to-face 
conversation.
This property is not true for email or when a centralized communications 
service is used.

\subsubsection{Standard Email}

The standard email system does not provide any security.
A suitable analogy would be that each message is a postcard, i.e.\ it has no 
envelope, so the content and address are visible on it.
This means that the postman can read the cards' contents, their recipients' and
senders' addresses.
(Yes, unlike real postcards these also include the sender's address.)
Furthermore, most postmen use transparent sacks to carry the postcards, so 
everyone along the way can also read the sender's and recipient's address and 
the contents.
However, some postmen have started using non-transparent sacks, i.e.\ encrypted 
connections between the servers, so those postcards can only be read by the 
staff in the post-office.
Thus the email system provides no confidentiality: each email server can read 
the messages, each network operator along the transport route can also read 
(and make a copy of) each email.
However, it is actually worse than that, because the email system provides no 
integrity either.
This mean that the postman, or anyone along the way, can do arbitrary 
modifications to the messages without anyone noticing the difference.
We can safely say that we cannot rely on the email system for neither security 
nor privacy when planning a protest.

When using a centralized communications service, such as Facebook, the level of
security and privacy we can achieve is that the postman carries non-transparent
sacks.
The business model of most such services is to read peoples postcards to better
profile their interests and thus deliver better suiting advertising.
This hides from third parties who is communicating with whom, but this 
information is available internally to the service.
This means that there are ways of learning this, through PRISM~\cite{Prism} for 
example.

\subsubsection{Secure Email and Text Messaging}

Secure email works by employing cryptography: we encrypt the contents of the 
postcard, providing confidentiality, and then add a digital signature to 
prevent modifications.
Thus the recipient is the only one who can read the message and the recipient 
can also verify that the message has not been modified along the way.
To make key management easy, most schemes use public-key cryptography.
This means that we have two keys, one which is public and another which is kept
private.
For encryption, the public key can transform a message to a ciphertext, i.e.\ 
a random-looking text string.
The private key can be used to transform the ciphertext back to the message.
Given only the public key, it is \enquote{impossible} to find the private key.
For signatures, we can use the private key to compute a signature of a message 
and then send the message and its signature.
The recipient can then use the public key to verify the signature of the 
message.
This signature depends on the entire message, so it is impossible to move 
a signature to another message --- as is possible with signatures on paper.
And since it is impossible to find the private key given only the public key, 
no one can create fake signatures.

The problem with the approach to secure email is that the digital signatures 
used provides a property called non-repudiation.
Say that Alice securely sent an email to Bob, if Eve would compromise Bob's 
private key, as many government agencies can, then she would learn that Alice 
--- and no one else --- has sent that message to Bob.
Bob might even give the message and his key to Eve voluntarily or under threat.
This is exactly the property that \citeauthor{otr2004} wanted to remove with 
\ac{OTR}.
They can do this by leveraging the interactive nature of \ac{IM} and changing 
the digital signatures to \acp{MAC}.
\Iac{MAC} is a shared-key construction, which means that Alice and Bob share 
the same key for generating and verifying \iac{MAC}.
This means that Bob can generate valid \acp{MAC} for any message and show to 
Eve, thus he cannot prove to Eve what Alice has said --- since he could have 
created this \enquote{proof} himself.
In addition, Alice and Bob do not use the same \ac{MAC} key throughout their 
conversation, then continuously exchange new keys, one for each message.
However, in this situation, Eve still has two candidates as the author of the 
message: Alice and Bob, since they both have access to the shared keys.
To remedy this problem Alice and Bob publishes the \ac{MAC} keys after use, 
i.e.\ when they no longer need them.
This gives everyone the possibility of generating messages that verifies under 
Alice and Bob's key, so now Alice and Bob can argue that someone (Eve included) 
could have modified the ciphertext.

The \ac{OTR} protocol became widely spread after the 2013 revelations about the
mass surveillance of the \ac{NSA} and \ac{GCHQ}, many derivatives of the 
protocol emerged in smartphone apps.
Among the most wide-spread derivatives of \ac{OTR} is Signal (formerly 
TextSecure)~\cite{SignalApp}.
The Signal protocol has, unlike many other of the derivatives, been formally 
analysed and proven that it indeed provides its claimed security 
properties~\cite{TextSecureAnalysis}.
One improvement over \ac{OTR} is the deniability.
In Signal the authentication is set up in such a way that any person knowing 
the public key of Alice and Bob can generate a fake transcript of 
a conversation.
This results in Eve having many more candidates for authoring a conversation.

\subsubsection{When the Adversary Controls the Network}

\subsubsection{When the Adversary Controls the Network}
\label{WhenAdversaryControlsNetwork}

\textcite{OTPKX} argue that if the adversary controls the entire network, then 
the approach to deniability taken by \ac{OTR} and Signal does not suffice.
The problem is that Eve can record a transcript of all communications
that have taken place.
We know that the \ac{NSA} did exactly that~\cite{XKeyscore} --- and more 
specifically, saved ciphertexts for later when the decryption key might be 
available.
In this setting it does not matter if anyone can generate a false transcript of 
a conversation between Alice and Bob, because Eve knows exactly what Alice has 
sent, what Bob has received and vice versa.
The argument of this class of protocol is that Alice and Bob have the 
possibility to deny anything about the conversation since it cannot be 
decrypted.
This seems extra problematic when even the free countries in the world suggest 
that there must be ways to break this 
encryption~\cite{BackDoorEncryption}\footnote{%
  We refer the reader to the text by \textcite{KeysUnderDoormats} for further 
  reasons for why this is a bad idea.
}.

There are more than one way to approach this problem.
The first approach would be to use an anonymizing service, such as 
Tor~\cite{Tor}.
This way, Eve would not know that Alice communicates with Bob, only that
Alice communicates with someone.
However, Alice and Bob are located in the same country and Eve controls the 
nationwide network.
For all low-latency anonymizing networks (such as Tor) where the entry point 
and exit are controlled by Eve, Eve can perform a time-correlation 
attack\footnote{%
  This means that Eve records the time of when each message enters the network 
  (entry distribution) and the time when each message exits the network (exit 
  distribution).
  Due to the low-latency property, these distributions will be related and Eve 
  can infer to whom Alice sent her message.
} and essentially render the anonymization service 
useless~\cite{SystemsForAnonymousCommunication}.
To make this attack more difficult for Eve, the system must introduce random 
delays in our communication\footnote{%
  The delays must transform the exit distribution to a distribution more 
  similar to the uniform distribution, then Eve's statistical analysis will 
  become more difficult.
}.
(We will return to this topic in \cref{MessageDistribution}.)
But despite all this, Eve can still ask Alice to decrypt the conversations, 
either she complies or claims that she does not know the key.

The second approach would be to ensure deniability even against this strong 
adversary.
This would not hide who communicates with whom, as in our first approach, but 
it provides deniability for the conversations.
The scheme suggested by \textcite{OTPKX} makes use of one practical instance of 
deniable encryption~\cite{DeniableEncryption}.
They construct a scheme where Alice and Bob can create \enquote{false proofs} 
for their conversation.
In essence, Eve records all traffic.
When she approaches Alice and asks her to provide a key to decrypt the recorded 
traffic, Alice can create a decryption key such that when Eve decrypts the 
recorded traffic will receive a plaintext of Alice's choice.
This way Alice can plausibly deny any allegations.
However, the question whether Eve would actually accept such a \enquote{proof}, 
knowing it might equally well be false, remains open.

