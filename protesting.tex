\documentclass[a4paper]{llncs}
\usepackage[utf8]{inputenc}
\usepackage[T1]{fontenc}
\usepackage[natbib,style=alphabetic]{biblatex}
\addbibresource{protesting.bib}

\title{%
  Future Technologies for\\
  Protests in the Information Age
}
\author{%
  Daniel Bosk
  \and
  the Group
}
\institute{%
  School of Computer Science and Communication,\\
  KTH Royal Institute of Technology,
  Stockholm\\
  \email{dbosk@kth.se}
}

\begin{document}
\maketitle

Nowadays, there is more data stored in online services than ever, mainly 
because of the widespread use of the Internet and the maturing of communication 
technologies, but also due to the popularization of Online Social Networks 
(OSNs).

Most of these services are run in a centralized manner such that the provider 
of the service acts as the communication channel between the users.
Being such a hub allows the providers to oversee a large amount of information, 
and in the case of OSNs, much of it of a personal and sensitive kind, e.g., 
life events, geolocated pictures or simply professional life.

OSNs have an unfortunate and controversial history of privacy issues, e.g., 
accidental and intentional data leakages, and security problems, e.g., 
censorship, discrimination.
Such dependence on the provider has gained relevance, and become a great 
concern for the general population in the recent years, especially after the 
news of the global mass surveillance program by the United States National 
Security Agency in collaboration with some of these providers and other
governmental intelligence agencies.

Users are fairly aware of the business model supporting, improving and 
maintaining these systems, i.e., advertisement or customer profiling.
However, these business models also create an incentive to keep collecting and 
mining more data, and a disincentive to protect the privacy of the user.

For such highly connected data-intensive services we propose a shift to 
decentralized and privacy-preserving solutions where users can have full 
control over their data.
Decentralized to reach provider independence, and privacy-preserving
to provide data protection by prevention, i.e., by cryptographic means
and access control.

Our proposal of a privacy-preserving decentralized OSN aims at implementing 
current functionality present in modern OSNs with privacy-preserving properties 
and exploring what other features can be implemented in such decentralized 
scenario.
We also analyse the trade-offs that have to be made in terms of meta-data 
inferences, heterogeneity, availability, scalability, robustness and 
efficiency.

We have designed some important building blocks: access control by applying 
a policy-hiding cryptographic scheme, usable password authentication in P2P 
networks, targeted user search by means of user defined knowledge threshold, 
and events coordination mechanism without the need of a trusted third party.
Currently, we are investigating the implications of aggregating the 
friend-activity feed, the scheduling of events in a privacy-preserving manner, 
and other efficient access control alternatives in our decentralized setting.


\printbibliography{}
\end{document}
