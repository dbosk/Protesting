\documentclass[a4paper]{llncs}
\usepackage[utf8]{inputenc}
\usepackage[T1]{fontenc}
\usepackage[natbib,style=alphabetic]{biblatex}
\addbibresource{surveillance.bib}

\usepackage[std]{libbib}

\title{%
  Future Technologies for\\
  Protests in the Information Age
}
\author{%
  Daniel Bosk
  \and
  the Group
}
\institute{%
  School of Computer Science and Communication,\\
  KTH Royal Institute of Technology,
  Stockholm\\
  \email{dbosk@kth.se}
}

\begin{document}
\maketitle

% XXX Types of protests
% - Arranging a real-world protest
% - Having an online protest
\begin{abstract}
  We propose a chapter on the possibilities of the future technologies that can 
  be used for protests.
  We focus on privacy-preserving tools that can be used before, under and after 
  a protest:
  \begin{itemize}
    \item before, to organize a public protest, either online or in the real 
      world;
    \item during, for the organizers and participants to communicate;
    \item after, possibly following up an event by verifying the participation 
      and computing verifiable statistics, e.g.\ how many participants and in 
      what area.
  \end{itemize}
  We give an overview of the current state-of-the-art and an outlook for the 
  future possibilities.
\end{abstract}


\section{Introduction}
% XXX The below doesn't connect with the above

Online technologies are used more and more by citizens all over the world.
This is both good and bad.
The bad thing is that there is more data stored in these online services than 
ever and these data are used by oppressive regimes against the citizens using 
them.
The good thing is that there are technologies which remedy some of the problems 
and researchers are trying to remedy the rest of the problems.

Most of these services are run in a centralized manner such that the provider 
of the service acts as the communication channel between the users.
Being such a hub allows the providers to oversee a large amount of information, 
and in the case of \acp{OSN}, much of it of a personal and sensitive kind, 
e.g.\ planning of events or geolocated pictures.
We have seen in the media of recent years how the security agencies also use 
these technologies: they tap into the data streams and monitor 
everybody~\cite{BoundlessInformant,XKeyscore} and even attack targets through 
these technologies~\cite{Quantum}.
Although not all nation-states' surveillance agencies have direct access to 
these data, there are other ways for them to achieve this detailed 
surveillance: e.g.\ the Great Firewall of China and similar systems set up by 
governments.
The result is that activists can be tracked down and jailed for posting 
critical comments about the ruling regime.

We can conclude from the above that we need stronger privacy properties for 
online technologies to protect citizens voicing their opinions in surveillance 
states.
One interesting branch of research is decentralized \acp{PET} or other 
privacy-preserving solutions where users can have full control over their data:
\begin{itemize}
  \item decentralized to reach provider independence and censorship resistance;
  \item privacy-preserving to provide data protection by prevention, e.g.\ by 
    cryptographic means.
\end{itemize}


\section{Before a Protest}
% XXX Write about mechanisms used before a protest

Events coordination mechanism without the need of \iac{TTP}.
Scheduling events in a privacy-preserving manner without the need for 
interaction.
We need these to be able to organize a protest.
We don't want to risk a oppressive regime killing the protest before it 
happens.
We also don't want the same regime to learn of who have committed to 
participate.


\section{During a Protest}
% XXX Write about mechanisms used during a protest

Privacy-preserving access-control for distributed storage.
This is interesting because we want to post photos and messages during the 
protest, e.g.\ photos of police brutality etc.
However, we don't want to be identifiable as a participating protester, as this 
might yield repercussions.


\section{After a Protest}
% XXX Write about mechanisms used after a protest
% - Verify the authenticity of the participation of a protest

We might want to verify the participation of a protest, not identifying 
individuals who participated, but we want to compute the number of participants 
to verify the scale of the protest.
This can be useful for e.g.\ the UN to verify protests happening in a country, 
then the UN can apply pressure.
At the moment, countries can deny the protests ever happened or at least the 
extent of the protest.

Purely online protests are essentially petitions.
This can be compared to electronic voting, to vote for or against something.
For this we should be able to use anonymous credentials, blind signatures and 
similar techniques, together with electronic voting protocols.

For real-world protests we should be able to use anonymous credentials, blind 
signatures and distance-bounding protocols with smartphones for the 
participants to build a network of verified participants.
Due to the distance-bounding protocol we can bind them all to having been at 
the same place.
With a time-stamping service we can make this a transitive relation, thus 
verifying the total number of participants of a real-world protest in 
a physical location.


\section{General Attacks against Activists}

Usable password authentication in \ac{P2P} networks.
Users need secure and usable authentication systems.
Protesters are usually ordinary people, not all rebellious people are 
cryptographers.

Targeted user search by means of user defined knowledge threshold.
Protesters must be able to find each other in the networks, but we don't want 
the oppressive regime to do the same.
This probably require some adaption and extension of the user-search mechanism.


\printbibliography{}
\end{document}
