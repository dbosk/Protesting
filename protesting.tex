\documentclass[a4paper]{llncs}
\usepackage[utf8]{inputenc}
\usepackage[T1]{fontenc}
\usepackage[defblank]{paralist}
\usepackage{cleveref}

\usepackage[natbib,style=alphabetic]{biblatex}
\addbibresource{crypto.bib}
\addbibresource{otrmsg.bib}
\addbibresource{ppes.bib}
\addbibresource{ac.bib}
\addbibresource{location.bib}
\addbibresource{reputation.bib}
\addbibresource{auth.bib}
\addbibresource{osn.bib}

\usepackage[crypto,std]{libbib}

\pagestyle{plain}

\title{%
  Future Technologies for\\
  Protests in the Information Age
}
\author{%
  Daniel Bosk
  \and
  Sonja Buchegger
}
\institute{%
  School of Computer Science and Communication,\\
  KTH Royal Institute of Technology,
  Stockholm\\
  \email{\{dbosk,buc\}@kth.se}
}

\begin{document}
\maketitle

\begin{abstract}
  We give an overview of the possible future technologies that can be used for 
  protests.
  We focus on privacy-preserving tools that can be used before, under and after 
  a protest:
  \begin{itemize}
    \item before, to organize a public protest, either online or in the real 
      world;
    \item during, for the organizers and participants to communicate, within 
      the group or to the outside world;
    \item after, possibly following up an event by verifying the participation 
      and computing verifiable statistics, e.g.\ how many participants and in 
      what area.
  \end{itemize}

  \keywords{%
    protesting;
    overview;
    privacy primitives;
    \acl{OSN}
  }
\end{abstract}


\section{Introduction}
\label{Introduction}
%Online technologies have become an essential part of millions of people all over the world. 
The rapid development of technology in the last couple decades of the 20\textsuperscript{th} 
century and its increasing prevalence in everyday life has helped society reach 
the 21\textsuperscript{st} century with a high standard of living, for example, 
real-time secure communications were a science-fiction notion a century ago while 
nowadays they are a popular means for communication for people and businesses. 
However, the success of such development has come with some trade offs, for example, 
in data collection. Better storage technologies have allowed for longer data retention 
policies for both the private and public sectors, not only to provide new and better 
services, but also to endanger the privacy of citizens, and sometimes their safety, 
usually in oppressive regimes.

Among these technological advances, \acp{OSN} stand out as a popular computer-mediated 
tool allowing people and other entities to interact by sharing and exchanging information 
of any kind. Computation power and network communication are combined to make social 
interactions between people possible at any time and in any place lessening political, 
economical and geographical boundaries.

Many \acp{OSN} are run in a centralized manner --- the service provider acts as 
a communication channel between the users of the \ac{OSN}. Such structure allows 
providers to oversee a large portion of the data, if not all, exchanged between 
the users. Bearing in mind that in the case of \acp{OSN}, much of it is of personal 
and sensitive kind, for example, posting a picture in the network may reveal the 
physical geolocation as this information can be embedded in the meta-data of the 
image.

Moreover, the massive collection of data in these networks makes them an ideal target 
for attackers such as competitors or even governmental agencies. For example, in 
the recent years, intelligence and security agencies of some countries have targeted 
these services to gain personal information about their citizens, enemies and even 
allies~\cite{Prism}.

While we acknowledge the benefits of such technological advances like \acp{OSN}, 
we also point out the costs to personal privacy and advocate for the need to develop 
privacy-enhancing technologies that can co-exist with these technologies. For example, 
decentralized solutions try to achieve provider independence and, in some cases, 
they also offer censorship resistance. Privacy-preserving solutions provide data 
protection by prevention, for example, by means of cryptographic techniques an organization 
could enforce certain policies instead of relying on the security of the system and its maintenance.

Despite the conventional and long-established form of protesting physically, online 
technologies have also opened the possibility to alternative ways, such as virtual 
\enquote{petitions}, or in general, expressing support for an opinion in the form 
of an encouraging comment or simply admiration. 

The more general case of expressing support for an opinion can be done in a hybrid 
manner using both physical contact and social media, such as \acp{OSN}. Petitions 
can be considered a particular case of electronic voting, a field of its own that 
is out of the scope of this chapter. 

In this chapter we focus on describing some privacy-enhancing tools in the context 
of \acp{OSN} that we believe can be useful in a protest and have not yet seen a 
widespread use in practice. Although a protest itself relies mainly in the traditional 
physical act of gathering, we believe that it would benefit from some of the developments 
originated in the fields of information security and privacy.

We categorize the topics of this chapter in a time-event manner in respect of the 
stages of a protest, namely: before, during and after.
\begin{itemize}
    \item \textbf{Before}\\
    Organization is the foremost task prior to the protest itself. 
    For example, decisions on the aim of the protest or the target audience that 
    is expected to participate in the protest.\\
    We take some of these issues in our scenario of \acp{OSN} in \cref{BeforeProtest}.

    \item \textbf{During}\\
    Communication during the protest is probably another essential 
    need for both the participants and the organizers of the protest. For example, 
    the organizers may need to get in touch with the press over the phone during 
    the protest.\\
    We discuss how these communications can be better protected in \cref{DuringProtest}.

    \item \textbf{After}\\
    Following up a protest is a relevant task for the organizers, not 
    only to assess their success but also to correct the flaws for the next time. 
    For example, the organizers may want to obtain reliable statistics on the number 
    of attendees per area.\\
    We discuss different authenticity and verifiability properties of use for this 
    stage in \cref{AfterProtest}.
\end{itemize}

Note that some of the tools and techniques we describe may be useful in more than 
one stage. Furthermore, there are some prerequisites for some of the tools that 
require either the protesters to perform some action in a stage prior to the one 
where the tool is actually used, for example, an invited participant must confirm 
the attendance before receiving details about the location of the protest.

\subsection{TODO: Give me a nice title}
% TODO Frame the following two problems. How do they fit (and help) in our scenario?
Before we begin our treatment of what was just outlined, there are two 
fundamental problems that we must discuss.
These are the problem of double agents and the problem of Sybil attacks.

\subsubsection{The Double Agent Problem}
\label{DoubleAgentProblem}

We cannot solve this problem, however, we might be able to reduce the damage 
\dots


\subsubsection{The Sybil Attack}
\label{SybilAttacks}

The Sybil attack is somewhat related to the double agent problem, but is only 
a problem in electronic systems.
The problem occurs when there is nothing that limits the creation of new 
identities, thus the adversary can create multiple unlinkable identities.
Usually the attack is aimed at reputation systems, where the adversary can use 
its many identities to vouch for each other to falsely gain in reputation.
In general, the problem can be summarized as that a rather small number of 
people in the network control a large part of the identities in the network in 
order to gain a disproportionate amount of influence~\cite{SybilAttack}.

\Citet{SybilAttack} proved that this problem cannot be solved without 
a \emph{logically} central control of the the creation of identities.
This means that we can not handle identity creation by letting the people who 
are already in the network vouch for other identities, i.e.\ to build a network 
of trust.
The Sybil attack itself aims at compromising a considerable portion of the 
identities, and thus, the more identities the attacker gains, the more new 
identities it can create and vouch for.
To prevent this kind of behaviour we rather need something like the national 
identification systems present in most countries, where the state has ensured 
a one-to-one correspondence between identities and physical persons.
Fortunately, there are techniques that can mitigate the effects of the Sybil 
attack without forcing us to use such a centralized identity system.
We will return to these where relevant in the text.
%TODO make sure we do

%This is, however, less desirable in the scenario we consider here.
%Fortunately, if we use techniques like \ac{MPC}, then we can overcome the 
%limitation of having a central authority.
%With \ac{MPC} we can construct a logical trusted third-party by having the 
%participants run a certain protocol, i.e.\ all participants constitute the.

%The Sybil attack problem remains unsolved without a central trusted broker that 
%acts as an identification authority, but there are methods and techniques in 
%the literature to mitigate its effect.
%We will return to these where it is relevant in the rest of the text.

% TODO Reprhase and complete the following
%In a large-scale scenario, like a decentralized network, is nearly impossible to 
%establish distinct identities without a central authority vouching for these identities.
%Entities can validate in two ways:
%Direct identity validation
%Indirect identity validation

%\dots



\section{Before a Protest}
\label{BeforeProtest}

There are several issues related to protesting in the stage before.
First comes the initial discussion between potential organizers, second comes 
the problem of scheduling this with participants.
For the initial discussion, the potential organizers don't want the regime's 
intelligence services to identify them as such.
One of the most popular secure-messaging protocols is \ac{OTR}, lately 
popularized in smartphones through Signal (formerly TextSecure).
\citet{OTPKX} proposed a scheme with the same properties as \ac{OTR} but with 
more deniability.
Another problem in this stage is to find other's user profiles.
\citet{ThresholdUserSearch} designed a scheme for targeted user search by means 
of user defined knowledge threshold.
Protesters must be able to find each other in the networks, but we don't want 
the oppressive regime to do the same.

For the scheduling of a protest, there are in turn several problems that must 
be addressed.
From the organizer Alice's perspective, she wants to protect herself from being 
arrested for organizing a protest.
So Alice needs to protect herself from the possible participants, as one of 
them can be agent Eve of the intelligence services of the regime.
From the participant Bob's perspective, he wants to protect himself from being 
arrested for committing to participate in a protest.
So Bob needs to protect himself from the organizer and the other participants, 
as any of them can be Eve.

When organizing a protest, what Alice and Bob want to agree on is a time, 
a place and to ensure enough people will show up at that time and place.
Alice and Bob also wants Eve to learn as little as possible of the plans so 
that she cannot curtail the protest.

\citet{EventsInvitations} presented a distributed protocol without the need of 
\iac{TTP}.
This protocols allows for different privacy settings:
\begin{itemize}
\item Alice discloses nothing to Bob, except the time and the place;
\item Alice discloses everything --- who the invitees are, who of those have 
  already committed etc.;
\item and every combination of settings in between.
\end{itemize}
Further, if Alice doesn't keep her promises, Bob has a proof which he can 
publish to everyone to show that Alice cheated.
Likewise, if Bob commits to attending, Alice has proof that Bob has done so and 
can show to everyone that Bob isn't present although he said he would.

One topic that must be explored is to adapt this protocol to introduce 
deniability.
Another interesting feature to include would be the choice of location.
With this, all participants can jointly agree on not only a time, but also 
a location.
This would help against the problem of announcing the location in advance.

%% EI contribution :: START
There are some tasks to accomplish prior to the protest itself that the organizers 
of the gathering need to arrange, for example, decide who are the most suitable 
candidates to attend the event, how to let them know about the protest and what 
preliminar information they should get or, later on, learn about the attendance of 
the invited ones in a privacy-preserving fashion.

However, realizing this standard feature of OSNs in a decentralized manner is not trivial 
because there is not a trusted third party to which both organizers and, invited and 
attending protestors can rely on. Because they all depend on themselves, an 
implementation of this feature must provide security properties that guarantee fairness 
to all parties involved, \eg a protestor can verify that the invitation she received 
was actually sent by the organizers. Moreover, the implementation should also provide 
with privacy settings to protect personal information such as the identities of 
the participants, for example, the organizers can restrict to only the invited participants 
to learn how many others have been invited, and only after a protestor has agreed 
and committed formally to attend the event, she can learn the identities of other 
invited protestors.

The challenge of implementing this feature without a trusted third party becomes 
greater when the organizers want to allow different types of information about the 
event to be shared with different groups of protestors in a secure way because any 
participant should be able to verify the results to detect any possible cheating. 
For example, a neutral trusted broker, such as the organizers, could keep certain 
information secret, such as the identities of the invited protestors, and only disclose 
it to those ones who commit to attend the protest. 

In our scheme, we describe and formalize a set of security and privacy properties 
to configure who can learn the identities of the invited or attending participants, 
or a more restrictive version in the count of invitees or attendees, and an attendee-only 
property that guarantees exclusive access to some information, for example, the 
location of the protest. By means of a set of privacy enhancing tools, such as storage 
location indirection, to control not only who can read some the contents of some 
encrypted information but also who can access the ciphertext; or the controlled 
ciphertext inference, to allow for controlled information leaks such as learning 
the number of invited participants but not their identities; and a commit-disclose 
protocol, to make some secret available for only those participants who committed 
to attend the protest while detecting at the same time any misbehaving party; we 
propose a trusted-third party free architecture, together with standard cryptographic 
primitives in our decentralized scenario.

%% EI contribution :: END


\section{During a Protest}
\label{DuringProtest}

There are a few aspects that we must cover which relate to this part of our 
treatment.
During a protest, organizers and demonstrators might want to communicate, 
either among themselves or to the outside world.
The communication to the outside world can have at least two purposes.
The first one is simply to try to get more people to come to the demonstration.
The second is when a demonstrator wants to store something for posterity.
This can be a photo capturing police brutality or a part of 
a proof-of-demonstration (explained in \cref{AfterProtest}).

There are a few problems with communication during a protest.
If the participants use the phone network, which generally is controlled by the 
government, they can be tracked and bound to the location by their phone.
If they communicate over the phone network they can still use the techniques 
outlined in \cref{BeforeProtest}.
However, if they don't want to be tracked, there are two options:
\begin{enumerate}
  \item they must use another network infrastructure that is not controlled by 
    the government,
  \item the mechanisms in \cref{BeforeProtest,AfterProtest} must allow 
    executions without access to a global communications infrastructure during 
    the demonstration.
\end{enumerate}

There are solutions to the first problem: wireless ad-hoc networks.
The area of ad-hoc networks is far too wide for us to convey more than the 
general idea of the field in this chapter.
The idea of ad-hoc networks is to form a network using ad-hoc connections.
For example, if Alice can communicate with Bob, Bob in turn can communicate 
with both Alice and Carol, then Alice can communicate with Carol through Bob.
Protesters can use this technique to form an ad-hoc network at the physical 
location of the demonstration, thus avoiding the government controlled network.
Depending on the reach of the ad-hoc network, participants might get access to 
the global Internet through some node in the network.
If not, they are limited to communicating only between themselves.


For the second problem listed above, we can achieve local communication, e.g.\ 
pair-wise communication through bluetooth or \ac{NFC}.
So we can assume that this possibility exists.
However, communicating certain data to the outside world will still be the 
ultimate goal, but we will use the local communication possibilities to ensure 
the authenticity of the data after the demonstration is over.
We will discuss this further in \cref{AfterProtest}.


\section{After a Protest}
\label{AfterProtest}

There are at least two things that are interesting after a protest:
\begin{inparaenum}[(a)]
\item that a third-party can verify the authenticity of the participation of 
  a protest;
\item for the organizers to use the participation as feedback into a reputation 
  system.
\end{inparaenum}
We want to verify the participation of a protest, but not identifying 
individuals who participated.
E.g.\ we want to compute the number of participants to verify the scale of the 
protest.
This can be useful for e.g.\ the UN to verify protests happening in a country; 
the country cannot deny it and the UN can apply pressure if needed.

Purely online protests are essentially petitions.
This can be compared to electronic voting --- to vote for or against something.
For this purpose we should be able to adapt existing electronic voting 
protocols.
In this case all participants and third-parties can verify the authenticity of 
the result, but cannot verify the vote of an individual.

For real-world protests we need to bind participants to the same physical 
location at a reasonably similar time (within the duration of the protest).
\citet{PROPS} developed a decentralized location-proof system which provides 
a participant with a proof of being at the location.
However, we would like that any participant can prove the demonstration's 
authenticity to a third-party, the scenario for this would be that only one 
demonstrator survives and manages to flee the country.

This location-proof system should be possible to combine with the scheduling 
system in \cref{BeforeProtest} for feedback into a reputation system.
One problem in these distributed systems is the possibility for the 
intelligence services to create multiple identities and thus perform a Sybil 
attack~\cite{SybilAttack} on the activists.




\section{General Attacks against Activists}
\label{GeneralAttacks}

In general, there are some more properties that are relevant for these systems.
Some secure authentication schemes allows for \ac{DoS} attacks against the 
proper account holder.
Although it prevents the attacker from gaining access, it also prevents the 
authentic user:
e.g.\ what should happen if multiple users try to access an account at the same 
time?
If the intelligence services can lock the activists out of their accounts and 
thus forcing them to resolve to less secure means of communication, then the 
intelligence services have won.
\citet{P2PPasswords} developed mechanisms towards solving this problem.


\printbibliography{}


\appendix
\section{Biography}
\label{Biography}
% XXX Write a <300 word biography
% http://www.gradhacker.org/2011/09/23/narrating-your-professional-life-writing-the-academic-bio/

Daniel Bosk is a doctoral student at KTH Royal Institute of Technology, 
Stockholm, Sweden, and a lecturer of Computer Engineering at Mid Sweden 
University, Sundsvall, Sweden.
He holds a Master of Science in Computer Science from KTH Royal Institute of 
Technology and a Master of Education in Mathematics and Computer Science from 
Stockholm University, Sweden.
His research interests are in security and privacy in decentralized systems, 
and specifically the empowerment of the users.

Guillermo Rodr\'{\i}guez-Cano is a doctoral student at KTH Royal Institute of 
Technology.
He holds a Master of Science in Computer Science from Uppsala University, 
Sweden, and a Bachelor of Engineering in Computer Science from University of 
Valladolid, Spain.
His research interests lie in the area of privacy and security in social 
systems and modelling, data mining, and information propagation in social 
networks.

Sonja Buchegger is an associate professor of Computer Science at KTH Royal 
Institute of Technology.
Prior to KTH, she was a
senior research scientist at Deutsche Telekom Laboratories in Berlin,
a post-doctoral scholar at the University of California at Berkeley,
and a researcher at the IBM Zurich Research Laboratory.
Her Ph.D.\ is in Communication Systems from EPFL, Lausanne, Switzerland, and 
she graduated in Computer Science from the University of Klagenfurt, Austria.
Her main research interests are in privacy and decentralized systems.

\end{document}
