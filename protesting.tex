\documentclass[a4paper]{llncs}
\usepackage[utf8]{inputenc}
\usepackage[T1]{fontenc}
\usepackage[defblank]{paralist}
\usepackage{cleveref}

\usepackage[natbib,style=alphabetic]{biblatex}
\addbibresource{crypto.bib}
\addbibresource{otrmsg.bib}
\addbibresource{ppes.bib}
\addbibresource{ac.bib}
\addbibresource{location.bib}
\addbibresource{reputation.bib}
\addbibresource{auth.bib}

\usepackage[crypto,std]{libbib}

\pagestyle{plain}

\title{%
  Future Technologies for\\
  Protests in the Information Age
}
\author{%
  Daniel Bosk
  \and
  Sonja Buchegger
}
\institute{%
  School of Computer Science and Communication,\\
  KTH Royal Institute of Technology,
  Stockholm\\
  \email{\{dbosk,buc\}@kth.se}
}

\begin{document}
\maketitle

\begin{abstract}
  We give an overview of the possible future technologies that can be used for 
  protests.
  We focus on privacy-preserving tools that can be used before, under and after 
  a protest:
  \begin{itemize}
    \item before, to organize a public protest, either online or in the real 
      world;
    \item during, for the organizers and participants to communicate, within 
      the group or to the outside world;
    \item after, possibly following up an event by verifying the participation 
      and computing verifiable statistics, e.g.\ how many participants and in 
      what area.
  \end{itemize}

  \keywords{%
    protesting;
    overview;
    privacy primitives;
    \acl{OSN}
  }
\end{abstract}


\section{Introduction}
\label{Introduction}

Online technologies are used more and more by citizens all over the world.
This is both good and bad.
The bad thing is that there is more data stored in these online services than 
ever and these data are used by oppressive regimes against the citizens using 
them.
The good thing is that there are technologies which remedy some of the problems 
and that researchers are trying to remedy the rest of the problems.

Most of these services are run in a centralized manner, i.e.\ the provider of 
the service acts as the communication channel between the users.
Being such a hub allows the providers to oversee a large amount of information, 
and in the case of \acp{OSN}, much of it is of personal and sensitive kind ---  
e.g.\ planning of events or geolocated pictures which reveal physical 
positions.
This also makes these providers the target of attackers who want to get that 
data.
We have seen in the media of recent years how the security agencies of many 
countries target these services to gain information about their citizens.

We can conclude that we need stronger privacy properties for online 
technologies to protect citizens in all countries.
One interesting branch of research is decentralized \acp{PET}:
\begin{itemize}
  \item decentralized or distributed solutions yield provider independence and 
    censorship resistance;
  \item privacy-preserving solutions provide data protection by prevention, 
    e.g.\ by cryptographic means rather than that some entity must maintain 
    a secure system.
\end{itemize}
In this chapter we propose to focus on privacy-preserving tools that can be 
used before, under or after a protest:
\begin{itemize}
  \item before, e.g.\ to organize a public protest, either online or in the 
    real world;
  \item during, e.g.\ for the organizers and participants to communicate, 
    within the group or to the outside world;
  \item after, e.g.\ possibly following up an event by verifying the 
    participation and computing verifiable statistics, e.g.\ how many 
    participants and in what area.
\end{itemize}


\subsection{Before a Protest}
\label{BeforeProtest}

There are several issues related to protesting in the stage before.
First comes the initial discussion between potential organizers, second comes 
the problem of scheduling this with participants.
For the initial discussion, the potential organizers don't want the regime's 
intelligence services to identify them as such.
One of the most popular secure-messaging protocols is \ac{OTR}, latelt 
popularized in smartphones through Signal (formerly TextSecure).
\citet{OTPKX} proposed a scheme with the same properties as \ac{OTR} but with 
more deniability.
Another problem in this stage is to find other's user profiles.
\citet{ThresholdUserSearch} designed a scheme for targeted user search by means 
of user defined knowledge threshold.
Protesters must be able to find each other in the networks, but we don't want 
the oppressive regime to do the same.

For the scheduling of a protest, there are in turn several problems that must 
be addressed.
From the organizer Alice's perspective, she wants to protect herself from being 
arrested for organizing a protest.
So Alice needs to protect herself from the possible participants, as one of 
them can be agent Eve of the intelligence services of the regime.
From the participant Bob's perspective, he wants to protect himself from being 
arrested for committing to participate in a protest.
So Bob needs to protect himself from the organizer and the other participants, 
as any of them can be Eve.

When organizing a protest, what Alice and Bob want to agree on is a time, 
a place and to ensure enough people will show up at that time and place.
Alice and Bob also wants Eve to learn as little as possible of the plans so 
that she cannot curtail the protest.

\citet{EventsInvitations} presented a distributed protocol without the need of 
\iac{TTP}.
This protocols allows for different privacy settings:
\begin{itemize}
\item Alice discloses nothing to Bob, except the time and the place;
\item Alice discloses everything --- who the invitees are, who of those have 
  already committed etc.;
\item and every combination of settings in between.
\end{itemize}
Further, if Alice doesn't keep her promises, Bob has a proof which he can 
publish to everyone to show that Alice cheated.
Likewise, if Bob commits to attending, Alice has proof that Bob has done so and 
can show to everyone that Bob isn't present although he said he would.

One topic that must be explored is to adapt this protocol to introduce 
deniability.
Another interesting feature to include would be the choice of location.
With this, all participants can jointly agree on not only a time, but also 
a location.
This would help against the problem of announcing the location in advance.

\subsection{During a Protest}
\label{DuringProtest}

During a protest, organizers and demonstrators might want to communicate, 
either among themselves or to the outside world.
There are a few problems with communication during a protest.
If the participants use the phone network, which generally is in the hands of 
the government, they can be tracked and bound to the location by their phone.
If they communicate over the phone network they can still use the techniques 
outlined in \cref{BeforeProtest}.
If they don't want to be tracked, they must use another network infrastructure 
that is not controlled by the government.

The communication to the outside world can have at least two purposes.
The first one is simply to try to get more people to come to the demonstration.
The second is when a demonstrator wants to store something for posterity.
This can be a photo capturing police brutality or a part of 
a proof-of-demonstration (explained in \cref{AfterProtest}).
\citet{DistStorAccessControl} presents work done in the area of 
privacy-preserving access control in distributed storage systems.
This is important since it outlines some possibilities and limits for such 
systems.
The problem in this scenario is that we don't want to be identifiable as 
a demonstrator, as this might yield repercussions.
Thus we can to share the data anonymously, no one can tell with whom we've 
shared what or if they've read it.
However, we still want to verify authenticity as it might otherwise be the 
regime spreading disinformation.

In this case it is also not straight forward to just apply the techniques in 
\cite{OTPKX} to achieve deniability.

\subsection{After a Protest}
\label{AfterProtest}

There are at least two things that are interesting after a protest:
\begin{inparaenum}[(a)]
\item that a third-party can verify the authenticity of the participation of 
  a protest;
\item for the organizers to use the participation as feedback into a reputation 
  system.
\end{inparaenum}
We want to verify the participation of a protest, but not identifying 
individuals who participated.
E.g.\ we want to compute the number of participants to verify the scale of the 
protest.
This can be useful for e.g.\ the UN to verify protests happening in a country; 
the country cannot deny it and the UN can apply pressure if needed.

Purely online protests are essentially petitions.
This can be compared to electronic voting --- to vote for or against something.
For this purpose we should be able to adapt existing electronic voting 
protocols.
In this case all participants and third-parties can verify the authenticity of 
the result, but cannot verify the vote of an individual.

For real-world protests we need to bind participants to the same physical 
location at a reasonably similar time (within the duration of the protest).
\citet{PROPS} developed a decentralized location-proof system which provides 
a participant with a proof of being at the location.
However, we would like that any participant can prove the demonstration's 
authenticity to a third-party, the scenario for this would be that only one 
demonstrator survives and manages to flee the country.

This location-proof system should be possible to combine with the scheduling 
system in \cref{BeforeProtest} for feedback into a reputation system.
One problem in these distributed systems is the possibility for the 
intelligence services to create multiple identities and thus perform a Sybil 
attack~\cite{SybilAttack} on the activists.

\subsection{General Attacks against Activists}
\label{GeneralAttacks}

In general, there are some more properties that are relevant for these systems.
Some secure authentication schemes allows for \ac{DoS} attacks against the 
proper account holder.
Although it prevents the attacker from gaining access, it also prevents the 
authentic user:
e.g.\ what should happen if multiple users try to access an account at the same 
time?
If the intelligence services can lock the activists out of their accounts and 
thus forcing them to resolve to less secure means of communication, then the 
intelligence services have won.
\citet{P2PPasswords} developed mechanisms towards solving this problem.


\printbibliography{}


\appendix
\section{Biography}
\label{Biography}
% XXX Write a <300 word biography
% http://www.gradhacker.org/2011/09/23/narrating-your-professional-life-writing-the-academic-bio/

Daniel Bosk is a doctoral student at KTH Royal Institute of Technology, 
Stockholm, Sweden, and a lecturer of Computer Engineering at Mid Sweden 
University, Sundsvall, Sweden.
He holds a Master of Science in Computer Science from KTH Royal Institute of 
Technology and a Master of Education in Mathematics and Computer Science from 
Stockholm University, Sweden.
His research interests are in security and privacy in decentralized systems, 
and specifically the empowerment of the users.

Guillermo Rodr\'{\i}guez-Cano is a doctoral student at KTH Royal Institute of 
Technology.
He holds a Master of Science in Computer Science from Uppsala University, 
Sweden, and a Bachelor of Engineering in Computer Science from University of 
Valladolid, Spain.
His research interests lie in the area of privacy and security in social 
systems and modelling, data mining, and information propagation in social 
networks.

Sonja Buchegger is an associate professor of Computer Science at KTH Royal 
Institute of Technology.
Prior to KTH, she was a
senior research scientist at Deutsche Telekom Laboratories in Berlin,
a post-doctoral scholar at the University of California at Berkeley,
and a researcher at the IBM Zurich Research Laboratory.
Her Ph.D.\ is in Communication Systems from EPFL, Lausanne, Switzerland, and 
she graduated in Computer Science from the University of Klagenfurt, Austria.
Her main research interests are in privacy and decentralized systems.

\end{document}
