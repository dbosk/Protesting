\documentclass[a4paper]{llncs}
\usepackage[utf8]{inputenc}
\usepackage[T1]{fontenc}
\usepackage[defblank]{paralist}
\usepackage{cleveref}

\usepackage[natbib,style=alphabetic]{biblatex}
\addbibresource{crypto.bib}
\addbibresource{otrmsg.bib}
\addbibresource{ppes.bib}
\addbibresource{ac.bib}
\addbibresource{location.bib}
\addbibresource{reputation.bib}
\addbibresource{auth.bib}
\addbibresource{privacy.bib}

\usepackage[crypto,std]{libbib}

\pagestyle{plain}

\title{%
  Future Technologies for\\
  Protests in the Information Age
}
\author{%
  Daniel Bosk
  \and
  Guillermo Rodr\'{\i}guez-Cano
  \and
  Sonja Buchegger
}
\institute{%
  Department of Theoretical Computer Science,\\
  School of Computer Science and Communication,\\
  KTH Royal Institute of Technology,
  Stockholm\\
  \email{\{dbosk,gurc,buc\}@kth.se}
}

\begin{document}
\maketitle


\section{Motivation}
\label{Motivation}
% 215 words

We propose a chapter on technologies that can be used for protests.
We focus on privacy-preserving tools that can be used before, during, and after 
a protest.
We give an overview of the current state-of-the-art and an outlook for future 
possibilities.

The book aims to analyse the use of \acp{ICT} for protests from the Arab Spring 
and onwards.
Our current research is on privacy-preserving protocols and mechanisms that 
can be used for protests.
Our work is in the domain of computer science, but we believe that it would 
be fruitful for the research communities interested in this area to have 
a chapter giving an overview of the technical possibilities.
We therefore propose to write such an overview, targeted at researchers without 
a background in computer science, to be included as e.g.\ a last chapter in the 
book.

We use the standard methodology of security research; namely to formalize the 
desired properties, formulate a theorem that says they hold and finally prove 
said theorem and provide at least a proof-of-concept implementation.
However, these details will not be included in the chapter, we will simply 
explain what these results mean to provide an outlook to the future 
possibilities for protests in the information age.
We will also set our work into context of related work by others.


\section{Outline}
\label{Outline}
% 434 words

Online technologies are used more and more by citizens all over the world.
This means there is more data stored in these online services than ever and 
these data are used by oppressive regimes against the citizens using them.
There are, however, technologies that remedy some of the problems
and researchers are working on the rest.

Most of these services are run in a centralized manner, i.e.\ the provider of 
the service acts as the communication channel between the users.
Being such a hub allows the providers to oversee a large amount of information, 
and in the case of \acp{OSN}, much of it is of personal and sensitive kind --- 
e.g.\ planning of events or geolocated pictures.
This also makes these providers the target of attackers who want to get that 
data and subject to government requests and subpoenas.
We have seen in the media of recent years how the security agencies of many 
countries target these services to gain information about their citizens.

To resolve this problem we need stronger privacy properties for online 
technologies, to protect citizens in all countries.
One promising branch of research is decentralized \acp{PET}:
\begin{itemize}
  \item decentralized or distributed solutions yield provider independence and 
    censorship resistance;
  \item privacy-preserving solutions provide data protection by prevention, 
    e.g.\ by cryptographic means rather than that someone must maintain 
    a secure system.
\end{itemize}
In this chapter we propose to focus on privacy-preserving tools that can be 
used before, under or after a protest:
\begin{description}

  \item[Before] E.g.\ for private discussion or to organize 
    a protest.
    For the private discussion we cover different aspects of deniability.
    For the organization of a public protest we explore different desirable 
    privacy properties, e.g.\ to protect the organizer from state-agents among 
    the participants and to protect the participants from a state-agent 
    organizer.

  \item[During] E.g.\ for the organizers and participants to communicate, 
    within the group or to the outside world.
    Communication to the outside world is interesting when a demonstrator wants 
    to store something for posterity (e.g.\ publish a photo of police brutality 
    or a proof-of-demonstration, see below).
    There are some problems here with tracking the publisher as well as 
    censorship.

  \item[After] E.g.\ for a third-party to verify a demonstration.
    By this we mean the possibility to compute \emph{verifiable} statistics, 
    e.g.\ how many participants and in what area.
    If one participant escapes the country, that participant should be able to 
    give proof of the demonstration to e.g.\ the UN\@.
    However, if a participant is apprehended by state-agents, that proof 
    shouldn't reveal anything about other participants.

\end{description}


\printbibliography{}


\appendix
\section{Biographies}
\label{Biography}
% http://www.gradhacker.org/2011/09/23/narrating-your-professional-life-writing-the-academic-bio/

Daniel Bosk is a doctoral student at KTH Royal Institute of Technology, 
Stockholm, Sweden, and a lecturer of Computer Engineering at Mid Sweden 
University, Sundsvall, Sweden.
He holds a Master of Science in Computer Science from KTH Royal Institute of 
Technology and a Master of Education in Mathematics and Computer Science from 
Stockholm University, Sweden.
His research interests are in privacy and decentralized systems, specifically 
to empower the users.

Guillermo Rodr\'{\i}guez-Cano is a doctoral student in the Department of 
Theoretical Computer Science at KTH Royal Institute of Technology since 2011.
He holds a Master of Science in Computer Science from Uppsala University since 
2010 and a Bachelor of Engineering in Computer Science from University of 
Valladolid since 2008.
His research interests lie in the area of privacy and security in social 
systems and modelling, data mining, and information propagation in social 
networks.

Sonja Buchegger is an associate professor of Computer Science at KTH Royal 
Institute of Technology, Stockholm, Sweden.
Prior to KTH, she was a
senior research scientist at Deutsche Telekom Laboratories in Berlin,
a post-doctoral scholar at the University of California at Berkeley,
and a researcher at the IBM Zurich Research Laboratory.
Her Ph.D.\ is in Communication Systems from EPFL, Lausanne, Switzerland, and 
she graduated in Computer Science from the University of Klagenfurt, Austria.
Her main research interests are in privacy and decentralized systems.

\end{document}
