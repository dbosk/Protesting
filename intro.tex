\section{Introduction}
\label{Introduction}
The rapid development of technology in the latter half of the
20th century and its increasing prevalence in
everyday life has helped large parts of the world to reach the
21st century with a means of having real-time secure
communications.  However, the success of such development has come
with some trade-offs, for example in data collection. Better storage
technologies have allowed for longer data retention policies for both
the private and public sectors, not only providing new and better
services to combat crime, but also compromising the privacy of
citizens, and sometimes their safety in oppressive regimes.

Among these technological advances, \acp{OSN} stand out as a popular
computer-mediated tool allowing people and other entities to interact
by sharing and exchanging information of any kind. Computation power
and network communication are combined to make social interactions
between people possible at any time and in any place lessening
political, economical and geographical boundaries. Such social media
are increasingly used for political activism ranging from showing
one's leanings by liking something to actual support and organization
of protests.

Many \acp{OSN} are run in a centralized manner --- the service
provider acts as a communication channel between the users of the
\ac{OSN}. Such structure allows providers to oversee a large portion
of the data, if not all, exchanged between the users. Bearing in mind
that in the case of \acp{OSN}, much of it is of personal and sensitive
kind, for example, posting a picture in the network may reveal the
physical geolocation as this information can be embedded in the
meta-data of the image. This has proved to be problematic for
political activism in several ways: a centralized ownership and
control make it easier to shut down, for example, the governmental block of Twitter in
Turkey for several days~\cite{TurkeyBansTwitter}, or public institutions
can subpoena information from the service provider. Moreover, the massive
collection of data in these networks makes them an ideal target for
attackers such as competitors or even governmental agencies. For
example, in recent years, intelligence and security agencies of
some countries have targeted these services to gain personal
information about their citizens, enemies and even
allies~\cite{Prism}. Since a centralized system can log not only data
the users upload but also meta-information about their behaviour, such
as online times, whom they communicate with, their location and social
ties, there is a wealth of information to connect a person to a cause.

While we acknowledge the benefits of such technological advances like
\acp{OSN}, we also point out the costs to personal privacy and
advocate for the need to develop \acp{PET} that
can co-exist with these technologies. For example, decentralized
solutions try to achieve provider independence and, in some cases,
they also offer censorship resistance. Privacy-preserving solutions
provide data protection by prevention, for example, by means of
cryptographic techniques an organization could enforce certain
policies instead of relying on the security of the system and its
maintenance.

Besides providing technological support to the conventional and
long-established form of protesting physically, online technologies
have also opened the possibility to alternative ways, such as virtual
\enquote{petitions}, or in general, expressing support for an opinion
in the form of an encouraging comment or simply affirmation.

In this chapter we focus on describing some privacy-enhancing tools in
the context of \acp{OSN} that we believe can be useful in a protest
and that have not yet seen a widespread use in practice. Although a
protest itself relies mainly on the traditional physical act of
gathering, we believe that it would benefit from some of the
developments originated in the fields of information security and
privacy.

\subsection{The Protesting Problem}

The topic of protesting is rather wide.
As suggested above, it could take the form of showing support for a statement 
in \iac{OSN}.
It could also take the form of people joining together in the streets for 
a demonstration.
For this chapter we will consider the following scenario:
Alice\footnote{We use the nomenclature of the Computer Security field,
  where Alice tries to securely communicate with Bob and others,
  so-called adversaries, interfere.} lives in a country under the rule of an authoritarian regime.
Alice wants to organize the opposition and lead a public protest to show that 
the people want a democratically reformed government.
As expected, the authoritarian regime wants to prevent this from happening.
The regime's goal is to oppress the opposition so that they cannot ever reach 
a big-enough protest to show the majority's dissatisfaction with the regime.
They will try to stop Alice as early as possible to avoid her ideas spread 
throughout the population.

Although our scenario is set in an authoritarian regime, the tools and
techniques described below are also useful in other systems of
governance, such as democracies --- because public protest and
demonstrations are of importance also in democracies, to keep them
democracies and not just to form them. More broadly, privacy
preservation is needed also in ostensible democracies, as illustrated
by whistleblowers such as Edward Snowden. A technical solution can
potentially prevent privacy breaches and thus enforce legal solutions.

There are, however, some limitations to what a technical solutions can achieve.
In particular we face two problems.
The first one, the double agent problem, is caused by humans ability to deceive 
each other, and consequently cannot easily be solved by technology.
The second one, the Sybil attack, is a consequence of the unlimited ability of creating 
arbitrary identities in online systems such as social networks.

\label{DoubleAgentProblem}
The double agent problem is the problem of one of the regime's agents 
infiltrating the opposition by acting as if part of the opposition.
We cannot solve this problem, however, we might be able to reduce the damage.
One design principle for privacy is data minimization, this strategy will help 
Alice reduce the information that the double agent, and anyone else, can learn.

\label{SybilAttacks}
The Sybil attack is somewhat related to the double agent problem, but is only 
a problem in electronic systems.
The problem occurs when there is nothing that limits the creation of new 
identities, thus the adversary can create multiple unlinkable identities.
Usually the attack is aimed at reputation systems, where the adversary can use 
its many identities to vouch for each other to falsely gain in reputation.
In general, the problem can be summarized as that a rather small number of 
people in the network control a large part of the identities in the network in 
order to gain a disproportionate amount of influence~\cite{SybilAttack}.

\Citet{SybilAttack} proved that this problem cannot be solved without 
a \emph{logically} central control of the the creation of identities.
This means that we can not handle identity creation by letting the people who 
are already in the network vouch for other identities, i.e.\ to build a network 
of trust.
The Sybil attack itself aims at compromising a considerable portion of the 
identities, and thus, the more identities the attacker gains, the more new 
identities it can create and vouch for.
To prevent this kind of behaviour we rather need something like the national 
identification systems present in most countries, where the state has ensured 
a one-to-one correspondence between identities and physical persons.
Fortunately, there are techniques that can mitigate the effects of the Sybil 
attack without forcing us to use such a centralized identity system.
We will return to these where relevant in the text.
%TODO make sure we do

%This is, however, less desirable in the scenario we consider here.
%Fortunately, if we use techniques like \ac{MPC}, then we can overcome the 
%limitation of having a central authority.
%With \ac{MPC} we can construct a logical trusted third-party by having the 
%participants run a certain protocol, i.e.\ all participants constitute the.

%The Sybil attack problem remains unsolved without a central trusted broker that 
%acts as an identification authority, but there are methods and techniques in 
%the literature to mitigate its effect.
%We will return to these where it is relevant in the rest of the text.

% TODO Reprhase and complete the following
%In a large-scale scenario, like a decentralized network, is nearly impossible to 
%establish distinct identities without a central authority vouching for these identities.
%Entities can validate in two ways:
%Direct identity validation
%Indirect identity validation

%\dots


\subsection{Outline}
\label{Outline}

In this chapter we will describe some privacy-enhancing tools that we have 
developed and that we believe are useful in the context of protesting.
We will describe them in relation to how they can be used, more specifically we 
categorize them as useful before, during or after a protest:
\begin{description}
  \item[Before]
    Organization, for example, decisions on the aim of the protest or the 
    target audience that is expected to participate in the protest.
    We address some of these issues in our scenario of \acp{OSN} in 
    \cref{BeforeProtest}.

  \item[During]
    Communication during the protest, for example, the organizers may need to 
    get in touch with the press over the phone during the protest.
    We discuss how these communications can be better protected in 
    \cref{DuringProtest}.

  \item[After]
    Following up a protest by the organizers, not only to assess their success 
    but also to correct the flaws for the next time.  For example, the 
    organizers may want to obtain reliable statistics on the number of 
    attendees per area.
    We discuss different authenticity and verifiability properties of use for 
    this stage in \cref{AfterProtest}.
\end{description}

