\section{Introduction}
\label{Introduction}

Online technologies are used more and more by citizens all over the world.
This is both good and bad.
The bad thing is that there are more data stored in these online services than 
ever and these data are used by oppressive regimes against the citizens using 
them.
The good thing is that there are technologies which remedy some of the problems 
and that researchers are trying to remedy the rest of the problems.

We will focus on the broad category of \acp{OSN} in this context.
Most of these services are today run in a centralized manner, i.e.\ the 
provider of the service acts as the communication channel between the users.
Such a hub structure allows the providers to oversee a large amount of 
information, and in the case of \acp{OSN}, much of it is of personal and 
sensitive kind ---  e.g.\ planning of events or geolocated pictures which 
reveal physical positions.
This also makes these providers the target of attackers who want to get those 
data:
as we have seen in the media of recent years, the security agencies of many 
countries also target these services to gain information about their 
citizens~\cite{Prism}.

We can conclude that we need strong privacy properties for online technologies 
to protect citizens in all countries of the world.
One interesting branch of research is decentralized \acp{PET}.
Decentralized solutions yield provider independence and censorship resistance.
Privacy-preserving solutions provide data protection by prevention, e.g.\ by 
cryptographic means rather than that some organization must maintain a secure 
system.

This chapter focuses on technical tools that can be useful for protesting.
We focus on privacy-preserving tools that exist in the research literature of 
the security and privacy field but have not yet seen widespread use in practice 
and some future research directions.
Protesting can be done physically, as we have known protests traditionally, 
they can also be done purely online in the form of \enquote{petitions}.
The latter is essentially the problem of electronic voting, we can transform 
the petition into a vote.
This would not be straight-forward and would include interesting research 
problems to be solved, however, we will focus this chapter on physical 
protests.

We categorize the topics we will cover as useful before, under or after 
a protest.
We will also treat the topics in that order, starting with tools useful before 
a protest and ending with tools useful after.
Some tools are of course useful both before and during a protest, we will point 
out how, but only cover the topic once.
Other tools that are useful afterwards requires us to perform some action 
during the protest, we will cover such a topic as a unit and simply point out 
what must be performed during and what must be done after the protest.
We will now give a short summarizing overview of the topics we will cover and 
which category they belong to.
Then, throughout the chapter, we will treat each topic in more details.

\subsection{Before a Protest}

E.g.\ to organize a public protest, we must be able to solve several problems.

\subsection{During a Protest}

During, e.g.\ for the organizers and participants to communicate, within the 
group or to the outside world;

\subsection{After a Protest}

After, e.g.\ possibly following up an event by verifying the participation and 
computing verifiable statistics, e.g.\ how many participants and in what area.

We have to deal with the problem of \emph{one} person being counted twice by 
signing the petition using \emph{two identities}.
This is generally known as the problem of Sybil attacks, and it has been proven 
impossible to solve without \emph{logically} central control of the creation of 
identities~\cite{SybilAttack}.

