\section{Introduction}
\label{Introduction}

Online technologies have become an essential part of the lives of millions of people 
all over the world. The increasing prevalence of technology has helped society reach 
the 21\textsuperscript{st} century with a high standard of living, for example, real-time 
secure communication was a science-fiction notion a century ago. However, the success 
of such development has come with some trade offs, for example in data collection, 
used by private and public entities not only to provide new services but also to 
endanger the privacy of citizens, and sometimes the safety, usually in oppressive 
regimes.

Among the technologies developed to date, \acp{OSN} are a popular application in 
the last couple of decades. Computation power and network communication are combined 
to make social interactions between people possible at any time and in any place 
lessening political, economical and geographical boundaries.

Many of the services 
offering a \ac{OSN} are run in a centralized manner --- the providers of the service 
act as a communication channel between the users of the service. Such structure 
allows the providers to oversee a large, if not all, portion of the data exchanged 
between the users. In the case of \acp{OSN}, much of it is of personal and sensititive 
kind, for example, posting a picture in the network may reveal the physical geolocation 
as it can be embedded in the meta-data of the image.

Moreover, the large collection of data in these networks makes them an ideal target 
for attackers such as competitors or even governmental agencies. For example, in 
the recent years, intelligence and security agencies of some countries have targeted 
these services to gain personal information about their citizens, enemies and even 
allies~\cite{Prism}.

We can conclude that we need strong privacy properties for online technologies 
to protect citizens in all countries of the world.
One interesting branch of research is decentralized \acp{PET}.
Decentralized solutions yield provider independence and censorship resistance.
Privacy-preserving solutions provide data protection by prevention, e.g.\ by 
cryptographic means rather than that some organization must maintain a secure 
system.

This chapter focuses on technical tools that can be useful for protesting.
We focus on privacy-preserving tools that exist in the research literature of 
the security and privacy field but have not yet seen widespread use in practice 
and some future research directions.
Protesting can be done physically, as we have known protests traditionally, 
they can also be done purely online in the form of \enquote{petitions} or, more 
generally, by expressing support for an expressed opinion.
The petitions can be seen as the problem of electronic voting, we can transform 
the petition into a vote.
This would not be straight-forward and would include interesting research 
problems to be solved, however, we will not discuss this in the chapter.
The more general case of expressing support for an expressed opinion is 
something that can be done using social media, which is more related to the 
\ac{OSN} mechanisms that we will discuss.
We will focus on the latter, and especially its use related to physical 
protests.

We categorize the topics we will cover as useful before, under or after 
a protest.
The before stage treats problems related to how to organize a public protest, 
this is discussed in \cref{BeforeProtest}.
The problems in the during the protest, e.g.\ how the organizers and 
participants can communicate securely within the group or to the outside world, 
is treated in \cref{DuringProtest}.
Finally, in the after stage we are interested in achieving different 
authenticity and verifiability properties, e.g.\ possibly following up an event 
by verifying the participation and computing verifiable statistics, such as how 
many participants and in what area.
This is treated in \cref{AfterProtest}.
Some tools are of course useful in more than one stage, so the categorization 
is not strict in that sense.
Furthermore, some tools that are useful afterwards requires us to perform some 
action during the protest, we will cover such a topic as a unit and simply 
point out what must be performed during and what must be done after the 
protest.

Before we begin our treatment of what was just outlined, there are two 
fundamental problems that we must discuss.
These are the problem of double agents and the problem of Sybil attacks.

\subsection{The Double Agent Problem}
\label{DoubleAgentProblem}

We cannot solve this problem, however, we might be able to reduce the damage 
\dots


\subsection{The Sybil Attack}
\label{SybilAttacks}

The Sybil attack is somewhat related to the double agent problem, but is only 
a problem in electronic systems.
The problem occurs when there is nothing that limits the creation of new 
identities, thus the adversary can create multiple unlinkable identities.
Usually the attack is aimed at reputation systems, where the adversary can use 
its many identities to vouch for each other to falsely gain in reputation.
In general, the problem can be summarized as that a rather small number of 
people in the network control a large part of the identities in the network in 
order to gain a disproportionate amount of influence~\cite{SybilAttack}.

\Citet{SybilAttack} proved that this problem cannot be solved without 
a \emph{logically} central control of the the creation of identities.
This means that we can not handle identity creation by letting the people who 
are already in the network vouch for other identities, i.e.\ to build a network 
of trust.
The Sybil attack itself aims at compromising a considerable portion of the 
identities, and thus, the more identities the attacker gains, the more new 
identities it can create and vouch for.
To prevent this kind of behaviour we rather need something like the national 
identification systems present in most countries, where the state has ensured 
a one-to-one correspondence between identities and physical persons.
Fortunately, there are techniques that can mitigate the effects of the Sybil 
attack without forcing us to use such a centralized identity system.
We will return to these where relevant in the text.
%TODO make sure we do

%This is, however, less desirable in the scenario we consider here.
%Fortunately, if we use techniques like \ac{MPC}, then we can overcome the 
%limitation of having a central authority.
%With \ac{MPC} we can construct a logical trusted third-party by having the 
%participants run a certain protocol, i.e.\ all participants constitute the.

%The Sybil attack problem remains unsolved without a central trusted broker that 
%acts as an identification authority, but there are methods and techniques in 
%the literature to mitigate its effect.
%We will return to these where it is relevant in the rest of the text.

% TODO Reprhase and complete the following
%In a large-scale scenario, like a decentralized network, is nearly impossible to 
%establish distinct identities without a central authority vouching for these identities.
%Entities can validate in two ways:
%Direct identity validation
%Indirect identity validation

%\dots


