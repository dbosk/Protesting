\section{Introduction}
\label{Introduction}
%Online technologies have become an essential part of millions of people all over the world. 
The rapid development of technology in the last couple decades of the 20\textsuperscript{th} 
century and its increasing prevalence in everyday life has helped society reach 
the 21\textsuperscript{st} century with a high standard of living, for example, 
real-time secure communications were a science-fiction notion a century ago while 
nowadays they are a popular means for communication for people and businesses. 
However, the success of such development has come with some trade offs, for example, 
in data collection. Better storage technologies have allowed for longer data retention 
policies for both the private and public sectors, not only to provide new and better 
services, but also to endanger the privacy of citizens, and sometimes their safety, 
usually in oppressive regimes.

Among these technological advances, \acp{OSN} stand out as a popular computer-mediated 
tool allowing people and other entities to interact by sharing and exchanging information 
of any kind. Computation power and network communication are combined to make social 
interactions between people possible at any time and in any place lessening political, 
economical and geographical boundaries.

Many \acp{OSN} are run in a centralized manner --- the service provider acts as 
a communication channel between the users of the \ac{OSN}. Such structure allows 
providers to oversee a large portion of the data, if not all, exchanged between 
the users. Bearing in mind that in the case of \acp{OSN}, much of it is of personal 
and sensitive kind, for example, posting a picture in the network may reveal the 
physical geolocation as this information can be embedded in the meta-data of the 
image.

Moreover, the massive collection of data in these networks makes them an ideal target 
for attackers such as competitors or even governmental agencies. For example, in 
the recent years, intelligence and security agencies of some countries have targeted 
these services to gain personal information about their citizens, enemies and even 
allies~\cite{Prism}.

While we acknowledge the benefits of such technological advances like \acp{OSN}, 
we also point out the costs to personal privacy and advocate for the need to develop 
privacy-enhancing technologies that can co-exist with these technologies. For example, 
decentralized solutions try to achieve provider independence and, in some cases, 
they also offer censorship resistance. Privacy-preserving solutions provide data 
protection by prevention, for example, by means of cryptographic techniques an organization 
could enforce certain policies instead of relying on the security of the system and its maintenance.

Despite the conventional and long-established form of protesting physically, online 
technologies have also opened the possibility to alternative ways, such as virtual 
\enquote{petitions}, or in general, expressing support for an opinion in the form 
of an encouraging comment or simply admiration. 

The more general case of expressing support for an opinion can be done in a hybrid 
manner using both physical contact and social media, such as \acp{OSN}. Petitions 
can be considered a particular case of electronic voting, a field of its own that 
is out of the scope of this chapter. 

In this chapter we focus on describing some privacy-enhancing tools in the context 
of \acp{OSN} that we believe can be useful in a protest and have not yet seen a 
widespread use in practice. Although a protest itself relies mainly in the traditional 
physical act of gathering, we believe that it would benefit from some of the developments 
originated in the fields of information security and privacy.

We categorize the topics of this chapter in a time-event manner in respect of the 
stages of a protest, namely: before, during and after.
\begin{description}
  \item[Before]
    Organization is the foremost task prior to the protest itself. 
    For example, decisions on the aim of the protest or the target audience 
    that is expected to participate in the protest.
    We take some of these issues in our scenario of \acp{OSN} in 
    \cref{BeforeProtest}.

  \item[During]
    Communication during the protest is probably another essential need for 
    both the participants and the organizers of the protest.
    For example, the organizers may need to get in touch with the press over 
    the phone during the protest.
    We discuss how these communications can be better protected in 
    \cref{DuringProtest}.

  \item[After]
    Following up a protest is a relevant task for the organizers, not only to 
    assess their success but also to correct the flaws for the next time.
    For example, the organizers may want to obtain reliable statistics on the 
    number of attendees per area.
    We discuss different authenticity and verifiability properties of use for 
    this stage in \cref{AfterProtest}.
\end{description}

Note that some of the tools and techniques we describe may be useful in more than 
one stage. Furthermore, there are some prerequisites for some of the tools that 
require either the protesters to perform some action in a stage prior to the one 
where the tool is actually used, for example, an invited participant must confirm 
the attendance before receiving details about the location of the protest.

\subsection{TODO: Give me a nice title}
% TODO Frame the following two problems. How do they fit (and help) in our scenario?
Before we begin our treatment of what was just outlined, there are two 
fundamental problems that we must discuss.
These are the problem of double agents and the problem of Sybil attacks.

\subsubsection{The Double Agent Problem}
\label{DoubleAgentProblem}

We cannot solve this problem, however, we might be able to reduce the damage 
\dots


\subsubsection{The Sybil Attack}
\label{SybilAttacks}

The Sybil attack is somewhat related to the double agent problem, but is only 
a problem in electronic systems.
The problem occurs when there is nothing that limits the creation of new 
identities, thus the adversary can create multiple unlinkable identities.
Usually the attack is aimed at reputation systems, where the adversary can use 
its many identities to vouch for each other to falsely gain in reputation.
In general, the problem can be summarized as that a rather small number of 
people in the network control a large part of the identities in the network in 
order to gain a disproportionate amount of influence~\cite{SybilAttack}.

\Citet{SybilAttack} proved that this problem cannot be solved without 
a \emph{logically} central control of the the creation of identities.
This means that we can not handle identity creation by letting the people who 
are already in the network vouch for other identities, i.e.\ to build a network 
of trust.
The Sybil attack itself aims at compromising a considerable portion of the 
identities, and thus, the more identities the attacker gains, the more new 
identities it can create and vouch for.
To prevent this kind of behaviour we rather need something like the national 
identification systems present in most countries, where the state has ensured 
a one-to-one correspondence between identities and physical persons.
Fortunately, there are techniques that can mitigate the effects of the Sybil 
attack without forcing us to use such a centralized identity system.
We will return to these where relevant in the text.
%TODO make sure we do

%This is, however, less desirable in the scenario we consider here.
%Fortunately, if we use techniques like \ac{MPC}, then we can overcome the 
%limitation of having a central authority.
%With \ac{MPC} we can construct a logical trusted third-party by having the 
%participants run a certain protocol, i.e.\ all participants constitute the.

%The Sybil attack problem remains unsolved without a central trusted broker that 
%acts as an identification authority, but there are methods and techniques in 
%the literature to mitigate its effect.
%We will return to these where it is relevant in the rest of the text.

% TODO Reprhase and complete the following
%In a large-scale scenario, like a decentralized network, is nearly impossible to 
%establish distinct identities without a central authority vouching for these identities.
%Entities can validate in two ways:
%Direct identity validation
%Indirect identity validation

%\dots


