\section{Introduction}
\label{Introduction}
The rapid development of information and communication technology (ICT) in the latter half of the
20th century and its increasing prevalence in
everyday life has helped a considerable part of the world to reach the
21st century with a means of having real-time secure
communications.  These developments, however,  have come
with some side effects, for example in data collection. Better storage
technologies have allowed for longer data retention policies for both
the private and public sectors, not only providing new and better
services to users, or means to combat crime, but also compromising the privacy of
citizens, and sometimes their safety in oppressive regimes.

Among these technological advances, \acp{OSN} (in the broadest sense,
any kind of online social media) stand out as a popular
computer-mediated tool allowing people and other entities to interact
by sharing and exchanging information of any kind. Computation power
and network communication are combined to make social interactions
between people possible at any time and in any place, lessening
political, economical and geographical boundaries. Such social media
are increasingly used for political activism ranging from showing
one's leanings by liking something to actual support and organization
of protests.

Many \acp{OSN} are run in a centralized manner --- the service
provider acts as a communication channel between the users of the
\ac{OSN}. Such structure allows providers to oversee a large portion
of the data, if not all, exchanged between the users. Bearing in mind
that in the case of \acp{OSN}, much of it is of personal and sensitive
kind, for example, posting a picture in the network may reveal the
physical geolocation of the poster as this information can be embedded
in the meta-data of the image or even inferred by image processing for
place recognition. This has proved to be problematic for political
activism in several ways: a centralized ownership and control make it
easier to shut down, for example, the governmental block of Twitter in
Turkey for several days~\cite{TurkeyBansTwitter}, or public
institutions can subpoena (at best) or just seize information from the
service provider. Moreover, the massive collection of data in these
networks makes them an ideal target for attackers such as competing
service providers or even governmental agencies. For example, in
recent years, intelligence and security agencies of some countries
have targeted these services to gain personal information about their
citizens, enemies and even allies~\cite{Prism}. Since a centralized
system can log not only data the users upload but also
meta-information about their behaviour, such as online times, whom
they communicate with, their location and social ties, there is a
wealth of information to connect a person to a cause.

While we acknowledge the benefits of such technological advances like
\acp{OSN}, we also point out the costs to personal privacy and
advocate for the need to develop \acp{PET} that can co-exist with
these technologies. A prominent example is Tor~\cite{Tor}, a routing
mechanism for online anonymity and censorship
resistance. Decentralized solutions try to achieve provider
independence and, in some cases, they also offer censorship
resistance. Privacy-preserving technologies aim at providing data
protection by prevention, i.e., making privacy breeches and data leaks
technically impossible. For example, by means of cryptographic
techniques, an organization could enforce certain policies instead of
relying on compliance by insiders and on the security of the system and its
maintenance to protect from intrusions. %TODO: good example with reference

Besides providing technological support to the conventional and
long-established form of protesting physically, online technologies
have also opened the possibility to alternative ways, such as virtual
\enquote{petitions}, or in general, expressing support for an opinion
in the form of an encouraging comment or simply affirmation.

In this chapter we focus on describing some privacy-enhancing tools in
the context of \acp{OSN} that we believe can be useful in a protest
and that have not yet seen a widespread use in practice. Although a
protest itself relies mainly on the traditional physical act of
gathering, we believe that it would benefit from some of the
developments originated in the fields of information security and
privacy.

\subsection{The Protesting Problem}

The topic of protesting is rather wide.
As suggested above, it could take the form of showing support for a statement 
in \iac{OSN}.
It could also take the form of people joining together in the streets for 
a demonstration.
For this chapter we will consider the following scenario:
Alice\footnote{%
  We use the nomenclature of the Computer Security field,
  where one user of the system (usually called Alice) tries to securely communicate
  with another (usually called Bob) and others (the so-called adversaries, usually
  named Eve) are trying to interfere, eavesdrop or otherwise attack the
  system. 

} lives in a country under the rule of an authoritarian regime.
Alice wants to organize the opposition and lead a public protest to show that 
the people want a democratically reformed government.
As expected, the authoritarian regime wants to prevent this from happening.
The regime's goal is to oppress the opposition so that they cannot ever reach 
a big-enough protest to show the majority's dissatisfaction with the regime.
They will try to stop Alice as early as possible to avoid her ideas spread 
throughout the population.

Although our scenario is set in an authoritarian regime, privacy
preservation is needed also in ostensible democracies, as illustrated
by whistle-blowers such as Edward Snowden. 
%, the tools and techniques described below are also useful in other systems of
%governance, such as democracies.  
A technical solution can potentially prevent privacy breaches and be
used to enforce data protection mandated by law as well as offer
additional privacy protection where such protection is insufficient.

While we have Alice's cause against a repressive non-democratic regime in mind 
when designing technologies like these, we must also consider that these 
technologies, just like any other technology, can be used by people with less admirable causes.
For example, these technologies might just as well be used by a fascist 
minority planning an attack against a democratic government.
As this type of tool benefits all people, it will inevitably also benefit those 
who want to coordinate actions against those in power --- be it the majority of 
the people in the case of democracy or a minority in the case of an 
authoritarian elite.
But if we consider that \SI{55}{\%} of the countries of the world are still 
non-free~\cite{FreedomInTheWorld2017}, we believe that any research in this 
direction will do more good than harm.

%According to \textcite{OurWorldInData-Terrorism} there were 13\,000 deaths 
%caused by terrorism in 2010.
%They compare this to deaths caused by interpersonal violence, 535\,000 cases
%in 2008, and road traffic accidents, 1\,209\,000 cases in 2008\footnote{%
%  In 2015, this number had increased to 1\,340\,000 deaths worldwide caused by 
%  road injuries~\cite{WHODeathStats}.
%}.
%We would thus prevent far more deaths by abolishing automobiles than we would 
%by preventing terrorism entirely --- including the terrorism planned using 
%\acp{PET}.


\subsection{Scope and Outline}
\label{Outline}

Over the years, we have developed privacy-preserving building blocks for 
\acp{DOSN}.
In this chapter we will describe how some of these can benefit the activist 
Alice in the scenario given above.
We will describe these privacy-enhancing tools in relation to how they can be 
used; more specifically, we categorize them as useful before, during or after 
a protest:
\begin{description}
  \item[Before]
    The first part is about organization, for example, decisions on the aim of the protest or the 
    target audience that is expected to participate in the protest.
    We address some of these issues in our scenario of \acp{OSN} in 
    \cref{BeforeProtest}.

  \item[During]
    Then we discuss communication during the protest, for example, the organizers may need to 
    get in touch with the press over the phone during the protest.
    We discuss how these communications can be better protected in 
    \cref{DuringProtest}.

  \item[After]
    Finally, we talk about following up a protest by the organizers, not only to assess their success 
    but also to correct the flaws for the next time.  For example, the 
    organizers may want to obtain reliable statistics on the number of 
    attendees per area, but protect that data from unauthorized
    entities, for example the police.
    We discuss different authenticity and verifiability properties of use for 
    this stage in \cref{AfterProtest}.
\end{description}

Before starting, however, we must provide a short discussion of some technical 
limitations.
