\section{Introduction}
\label{Introduction}

Online technologies are used more and more by citizens all over the world.
This is both good and bad.
The bad thing is that there are more data stored in these online services than 
ever and these data are used by oppressive regimes against the citizens using 
them.
The good thing is that there are technologies which remedy some of the problems 
and that researchers are trying to remedy the rest of the problems.

We will focus on \acp{OSN} in this context.
Most of these services are today run in a centralized manner, i.e.\ the 
provider of the service acts as the communication channel between the users.
Being such a hub allows the providers to oversee a large amount of information, 
and in the case of \acp{OSN}, much of it is of personal and sensitive kind ---  
e.g.\ planning of events or geolocated pictures which reveal physical 
positions.
This also makes these providers the target of attackers who want to get those 
data.
As we have seen in the media of recent years, the security agencies of many 
countries also target these services to gain information about their citizens.

We can conclude that we need stronger privacy properties for online 
technologies to protect citizens in all countries of the world.
In this chapter we focus on privacy-preserving tools that exist in the research 
literature of the security and privacy field but have not yet seen widespread 
use in practice.
We categorize them as useful before, under or after a protest:
\begin{itemize}
  \item before, e.g.\ to organize a public protest, either online or in the 
    real world;
  \item during, e.g.\ for the organizers and participants to communicate, 
    within the group or to the outside world;
  \item after, e.g.\ possibly following up an event by verifying the 
    participation and computing verifiable statistics, e.g.\ how many 
    participants and in what area.
\end{itemize}

One interesting branch of research is decentralized \acp{PET}:
\begin{itemize}
  \item decentralized or distributed solutions yield provider independence and 
    censorship resistance;
  \item privacy-preserving solutions provide data protection by prevention, 
    e.g.\ by cryptographic means rather than that some entity must maintain 
    a secure system.
\end{itemize}

